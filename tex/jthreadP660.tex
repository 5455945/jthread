%!TEX root = std.tex
\makeevenhead{cpppage}{Josuttis, Sutter, Williams: \tcode{jthread}}{}{\textbf{P0660R4}}
\makeoddhead{cpppage}{Josuttis, Sutter, Williams: \tcode{jthread}}{}{\textbf{P0660R4}}

{\small
\begin{tabular}{@{}ll}
Project:  	& ISO JTC1/SC22/WG21: Programming Language C++ \\
Doc No: 	& WG21 P0660R4 \\
Date: 		& 2018-09-11 \\
Reply to: 	& Nicolai Josuttis (nico@josuttis.de), Herb Sutter (hsutter@microsoft.com), \\
                &         Anthony Williams (anthony@justsoftwaresolutions.co.uk) \\
Audience: 	& SG1, LEWG, LWG \\
Prev. Version:	& \url{www.wg21.link/P0660R3} \\
\end{tabular}
}

\section*{{\huge{}A Cooperatively Interruptible Joining Thread, Rev 4}}

\subsubsection*{New in R4}
\begin{itemize}
 \item Removed interruptible CV waiting members that don't take a predicate.
 \item Removed adding a new \tcode{cv_status} value \tcode{interrupted}.
 \item Added CV members for interruptible timed waits.
 \item Renamed CV members that wait interruptible.
\end{itemize}

\subsection*{Purpose}

This is the proposed wording for a cooperatively interruptible joining thread.

For a full discussion fo the motivation, see
\url{www.wg21.link/p0660r0} and
\url{www.wg21.link/p0660r1}.

A default implementation exists at:
\url{github.com/josuttis/jthread}.
Note that the proposed functionality can be fully implemented 
on top of the existing C++ standard library.

Basis examples:
\begin{codeblock}
void testJThread() 
{
  std::jthread t([] {
                   while (...) {
                     std:this_thread::throw_if_interrupted();
                     ...
                     // temporarily disable interrupts:
                     auto origToken = std::this_thread::exchange_interrupt_token(std::interrupt_token{});
                     ...
                     std::this_thread::exchange_interrupt_token(origToken);
                     ...
                   }
                 });
  ...
} // jthread destructor signals interrupt and therefore silently ends the started thread
\end{codeblock}

\begin{codeblock}
void testInterruptibleCVWait() 
{
  bool ready = false;
  std::mutex readyMutex;
  std::condition_variable readyCV;
  std::jthread t([&ready, &readyMutex, &readyCV] {
                    while (...) {
                      ...
                      {
                        std::unique_lock lg{readyMutex};
                        readyCV.iwait_for(lg,
                                          std::chrono::seconds(2),
                                          [&ready] {
                                             return ready;
                                          });
                      }
                      ...
                    }
                  });
  ...
} // jthread destructor signals interrupt and therefore unblocks the CV wait and silently ends the started thread
\end{codeblock}

\subsubsection*{Feature Test Macro}

This is a new feature so that it shall have the following feature macro:
\begin{codeblock}
	__cpp_lib_jthread
\end{codeblock}


\subsubsection*{Acknowledgements}

Thanks to all who incredibly helped me to prepare this paper, such as all people in the C++ concurrency and library working group.
Especially, we want to thank: Hans Boehm, Olivier Giroux, Pablo Halpern, Howard Hinnant, Alisdair Meredith, Gor Nishanov, Ville Voutilainen, and Jonathan Wakely.


\section*{Proposed Wording}
All against N4659.

{\color{blue}
[{\itshape{}Editorial note:} This proposal uses the LaTeX macros of the draft standard.
        To adopt it please ask for the LaTeX source code of the proposed wording. ]
}

\clearpage

