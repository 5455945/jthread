%!TEX root = std.tex
\makeevenhead{cpppage}{Josuttis, Baker, O'Neil, Sutter, Williams: \tcode{jthread}}{}{\textbf{P0660R7}}
\makeoddhead{cpppage}{Josuttis, Baker, O'Neil, Sutter, Williams: \tcode{jthread}}{}{\textbf{P0660R7}}

{\small
\begin{tabular}{@{}ll}
Project:  	& ISO JTC1/SC22/WG21: Programming Language C++ \\
Doc No: 	& WG21 P0660R8 \\
Date: 		& 2019-01-13 \\
Reply to: 	& Nicolai Josuttis (nico@josuttis.de), \\
                &         Lewis Baker (lbaker@fb.com) \\
                &         Billy O'Neal (bion@microsoft.com) \\
                &         Herb Sutter (hsutter@microsoft.com), \\
                &         Anthony Williams (anthony@justsoftwaresolutions.co.uk) \\
Audience: 	& SG1, LEWG, LWG \\
Prev. Version:	& \url{www.wg21.link/P0660R7}, \url{www.wg21.link/P1287R0} \\
\end{tabular}
}

%******************************************************************
\section*{{\huge{}Stop Tokens and a Joining Thread, Rev 8}}

\subsubsection*{New in R8}
As requested at
\href{http://wiki.edg.com/bin/view/Wg21sandiego2018/P0660}{the LEWG meeting in San Diego 2018}:
\begin{itemize}
 \item Terminology (especially rename \tcode{interrupt_token} to \tcode{stop_token}.
 \item Add a deduction guide for \tcode{stop_callback}
 \item Add \tcode{std::nostopstate_t} to create stop stop tokens that don't share a stop state
 \item Several clarifications in wording
\end{itemize}

\subsubsection*{New in R7}
\begin{itemize}
 \item Adopt \url{www.wg21.link/P1287} as discussed in
        \href{http://wiki.edg.com/bin/view/Wg21sandiego2018/P1287R0}{the SG1 meeting in San Diego 2018},
        which includes:
        \begin{itemize}
         \item Add callbacks for interrupt tokens.
         \item Split into \tcode{interrupt_token} and \tcode{interrupt_source}.
        \end{itemize}
\end{itemize}

\subsubsection*{New in R6}
\begin{itemize}
 \item User \tcode{condition_variable_any} instead of \tcode{consition_variable}
        to avoid all possible races, deadlocks, and unintended undefined behavior.
 \item Clarify future binary compatibility for interrupt handling
        (mention requirements for future callback support and allow \tcode{bad_alloc} exceptions on waits.
\end{itemize}

\subsubsection*{New in R5}
As requested at
\href{http://wiki.edg.com/bin/view/ExecSeattle2018/MinutesDay2}{the SG1 meeting in Seattle 2018}:
\begin{itemize}
 \item Removed exception class \tcode{std::interrupted} and the \tcode{throw_if_interrupted()} API.
 \item Removed all TLS extensions and extensions to \tcode{std::this_thread}.
 \item Added support to let \tcode{jhread} call a callable that
        either takes the interrupt token as additional first argument
        or doesn't get it (taking just all passed arguments).
\end{itemize}

\subsubsection*{New in R4}
\begin{itemize}
 \item Removed interruptible CV waiting members that don't take a predicate.
 \item Removed adding a new \tcode{cv_status} value \tcode{interrupted}.
 \item Added CV members for interruptible timed waits.
 \item Renamed CV members that wait interruptible.
 \item Several minor fixes (e.g. on \tcode{noexcept}) and full proposed wording.
\end{itemize}

%**************************
\subsection*{Purpose}

This is the proposed wording for a cooperatively interruptible joining thread.

For a full discussion fo the motivation, see
\url{www.wg21.link/p0660r0} and
\url{www.wg21.link/p0660r1}.

A default implementation exists at:
\url{http://github.com/josuttis/jthread}.
Note that the proposed functionality can be fully implemented 
on top of the existing C++ standard library without special OS support.

%**************************
\subsection*{Basis examples}

\begin{itemize}
 \item At the end of its lifetime a \tcode{jthread} automatically signals a request to stop the started thread
        (if still joinable) and joins:
\begin{codeblock}
void testJThreadWithToken() 
{
    std::jthread t([] (std::stop_token stoken) {
                     while (!stoken.stop_requested()) {
                       //...
                     }
                   });
    //...
} // jthread destructor signals requests to stop and therefore ends the started thread and joins
\end{codeblock}

The stop could also be explicitly requested with \tcode{t.request_stop()}.

 \item If the started thread doesn't take an stop token, 
        the destructor still has the benefit of calling \tcode{join()} (if still joinable):
\begin{codeblock}
void testJThreadJoining()
{
    std::jthread t([] {
                     //...
                   });
    //...
} // jthread destructor calls join()
\end{codeblock}
        This is a significant improvement over \tcode{std::thread} where you had to program the following
        to get the same behavior (which is common in many scenarios):
\begin{codeblock}
void compareWithStdThreadJoining()
{
    std::thread t([] {
                    //...
                  });
    try {
      //...
    }
    catch (...) {
      j.join();
      throw;  // rethrow
    }
    t.join();
}
\end{codeblock}

 \item An extended CV API enables to interrupt CV waits using the passed stop token
        (i.e. interrupting the CV wait without polling):
\begin{codeblock}
void testInterruptibleCVWait() 
{
  bool ready = false;
  std::mutex readyMutex;
  std::condition_variable_any readyCV;
  std::jthread t([&ready, &readyMutex, &readyCV] (std::stop_token st) {
                    while (...) {
                      ...
                      {
                        std::unique_lock lg{readyMutex};
                        readyCV.wait_until(lg,
                                           [&ready] {
                                              return ready;
                                           },
                                           st);  // also ends wait on stop request for \tcode{st}
                      }
                      ...
                    }
                  });
  ...
} // jthread destructor signals stop request and therefore unblocks the CV wait and ends the started thread
\end{codeblock}
\end{itemize}

%**************************
\subsection*{Feature Test Macro}

This is a new feature so that it shall have the following feature macro:
\begin{codeblock}
	__cpp_lib_jthread
\end{codeblock}


%**************************
\subsection*{Key Design Hints}

\subsubsection*{Guarantees for Races}

\subsubsection*{Other Hints}

The terminology was carfully selected with the following reasons
\begin{itemize}
 \item With a stop token we neither "interrupt" nor "cancel" something.
        We request a stop that cooperatively has to get handled.
 \item \tcode{stop_possible()} helps to avoid addign new callbacks
        or checking for stop states.
        The name was selected to have a common and pretty self-explanatory name
        that is shared by both \tcode{stop_source}s and \tcode{stop_token}s.
\end{itemize}

The deduction guide for \tcode{stop_callback}s enables
constructing a \tcode{stop_callback} with an lvalue callable:
\begin{codeblock}
auto lambda = []{};
std::stop_callback cb{ token, lambda };  // captures by reference
\end{codeblock}

Adding a new callback is \tcode{noexcept} (unless moving the passed function throws).


%**************************
\subsection*{Acknowledgements}

Thanks to all who incredibly helped me to prepare this paper, such as all people in the C++ concurrency and library working group.
Especially, we want to thank:
 Hans Boehm, Olivier Giroux, Pablo Halpern, Howard Hinnant, Alisdair Meredith, Gor Nishanov,
 Tony Van Eerd, Ville Voutilainen, and Jonathan Wakely.


%******************************************************************
\section*{Proposed Wording}
All against N4762.

{\color{blue}
[{\itshape{}Editorial note:} This proposal uses the LaTeX macros of the draft standard.
        To adopt it please ask for the LaTeX source code of the proposed wording. ]
}

\clearpage

