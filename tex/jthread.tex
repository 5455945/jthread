%!TEX root = std.tex
\rSec0[thread]{Thread support library}

%******************************************************************
\rSec1[jthread.general]{General}

\pnum
The following subclauses describe components to create and manage
threads\iref{intro.multithread}, perform mutual exclusion, and communicate conditions
and values
between threads, as summarized in \tref{thread.lib.summary}.

\begin{libsumtab}{Thread support library summary}{tab:thread.lib.summary}
\ref{thread.req}      & Requirements          &                               \\ \rowsep
\ref{thread.threads}  & Threads               & \tcode{<thread>}              \\ \rowsep
\color{insertcolor}
\ref{thread.interrupt_token} &
        \color{insertcolor} Interrupt Tokens       &
                \color{insertcolor} \tcode{<interrupt_token>}              \\ \rowsep
\color{insertcolor}
\ref{thread.jthreads} &
        \color{insertcolor} Joining Threads       &
                \color{insertcolor} \tcode{<jthread>}              \\ \rowsep
\ref{thread.mutex}    & Mutual exclusion      & \tcode{<mutex>}               \\
                      &                       & \tcode{<shared_mutex>}        \\ \rowsep
\ref{thread.condition}& Condition variables   & \tcode{<condition_variable>}  \\ \rowsep
\ref{futures}         & Futures               & \tcode{<future>}              \\
\end{libsumtab}

%******************************************************************
\rSec1[thread.req]{Requirements}


%******************************************************************
\rSec1[thread.threads]{Threads}


\clearpage

{\color{insertcolor}

%******************************************************************
\rSec1[thread.interrupt_token]{Interrupt Tokens}

\pnum
\ref{thread.interrupt_token} describes components that can be used to
asynchonously signal an interrupt.
The interrupt can only be signaled once.

%**************************
\rSec2[thread.interrupt_token.syn]{Header \tcode{<interrupt_token>} synopsis}
\indexhdr{interrupt_token}%

\begin{codeblock}
namespace std {
  // \ref{interrupted} class \tcode{interrupted}
  class interrupted;

  // \ref{thread.interrupt_token.syn} class \tcode{interrupt_token}
  class interrupt_token;
}
\end{codeblock}

%**************************
\indexlibrary{\idxcode{interrupted}}%
\rSec2[interrupted]{Class \tcode{interrupted}}

\indexlibrary{\idxcode{exception}}%
\begin{codeblock}
namespace std {
  class interrupted {
  public:
    explicit interrupted() noexcept;
    explicit interrupted(const interrupted&) noexcept;
    interrupted& operator=(const interrupted&) noexcept;
    const char* what() const noexcept;
  };
}
\end{codeblock}

\pnum
The class
\tcode{interrupted}
defines the type of objects thrown as
exceptions by \Cpp{} standard library components,
and certain expressions, to report a signaled interrupt.
\begin{note} This class is not derived from class \tcode{exception}. \end{note}

\indexlibrary{\idxcode{interrupted}!constructor}%
\begin{itemdecl}
interrupted() noexcept;
\end{itemdecl}

\begin{itemdescr}
\pnum
\effects
Constructs an object of class
\tcode{interrupted}.
\end{itemdescr}

\indexlibrary{\idxcode{interrupted}!constructor}%
\indexlibrarymember{operator=}{interrupted}%
\begin{itemdecl}
interrupted(const interrupted&) noexcept;
interrupted& operator=(const interrupted&) noexcept;
\end{itemdecl}
\begin{itemdescr}
  \pnum \effects Copies an object of class \tcode{interrupted}.

  \pnum \postconditions If \tcode{*this} and \tcode{rhs} both have dynamic type \tcode{interrupted}
        then the value of the expression \tcode{strcmp(what(), rhs.what())} shall equal 0.
\end{itemdescr}

\indexlibrarymember{what}{interrupted}%
\begin{itemdecl}
const char* what() const noexcept;
\end{itemdecl}
% NOTE: NO override here

\begin{itemdescr}
\pnum
\returns
An \impldef{return value of \tcode{interrupted::what}} \ntbs{}.

\pnum
\remarks
The message may be a null-terminated multibyte string\iref{multibyte.strings},
suitable for conversion and display as a
\tcode{wstring}~(\ref{string.classes}, \ref{locale.codecvt}).
The return value remains valid until the exception object from which
it is obtained is destroyed or a non-\tcode{const}
member function of the exception object is called.
\end{itemdescr}


%**************************
\indexlibrary{\idxcode{interrupt_token}}%
\rSec3[interrupt_token]{Class \tcode{interrupt_token}}

\pnum
\indexlibrary{\idxcode{interrupt_token}}%
The class \tcode{interrupt_token} implements semantics of shared ownership of a 
token to signal interrupts.
An interrupt can only be signaled once.
All owners can signal an interrupt, provided the token is valid.
All owners can check whether an interrupt was signaled.
The last remaining owner of the token is responsible for destroying
the object.

\begin{codeblock}
namespace std {
  class interrupt_token {
  public:
    explicit interrupt_token() noexcept;
    explicit interrupt_token(bool initial_state);

    interrupt_token(const interrupt_token&) noexcept;
    interrupt_token(interrupt_token&&) noexcept;
    interrupt_token& operator=(const interrupt_token&) noexcept;
    interrupt_token& operator=(interrupt_token&&) noexcept;
    void swap(interrupt_token&) noexcept;

    // \ref{interrupt_token.mem} interrupt handling:
    bool valid() const noexcept;
    bool is_interrupted() const noexcept;
    bool interrupt();
    void throw_if_interrupted() const;
  }
}

bool operator== (const interrupt_token& lhs, const interrupt_token& rhs);
bool operator!= (const interrupt_token& lhs, const interrupt_token& rhs);
\end{codeblock}

Calls to \tcode{interrupt()}, \tcode{is_interrupted()}, and \tcode{throw_if_interrupted()}
are atomic operations(6.8.2.1p3 \ref{intro.races}) on an atomic object contained in the interrupt_token.
Hence concurrent calls to these functions do not introduce data races. 
A call to \tcode{interrupt()} synchronizes with any call to \tcode{interrupt()},
\tcode{is_interrupted()}, or \tcode{throw_if_interrupted()} that observes the interrupt
(and hence returns \tcode{true} or throws).

% Hans Boehm:
%The fact that they are atomic operations implies that they appear to be be executed in a single total per-interrupt-token order. We could instead require them to be sequentially consistent, and participate in the SC order. But I think this is another one of those cases in which only clearly silly code can tell the difference. And we've generally not insisted on seq_cst unless there is some real possibility of confusion. (You can implement Dekker's algorithm with interrupt_tokens. But if you do, your code deserves to break.)
%
%I'm using "synchronizes with" instead of "strongly happens before" since I'm calling these atomic operations. Since there are no sequentially consistent operations on the atomic, I believe the difference is not observable, even after the memory model fixes.
%
% Hans Boehm:
%We then also need some statement for the waiting functions that if they return with an interrupted status, their synchronization behavior is as though they called is_interrupted().


\rSec3[interrupt_token.constr]{\tcode{interrupt_token} constructors}

\indexlibrary{\idxcode{interrupt_token}!constructor}%
\begin{itemdecl}
interrupt_token() noexcept;
\end{itemdecl}
\begin{itemdescr}
  \pnum\effects Constructs a new \tcode{interrupt_token} object that can't be used to signal interrupts.

  \pnum\postconditions \tcode{valid() == false}.
\end{itemdescr}

\indexlibrary{\idxcode{interrupt_token}!constructor}%
\begin{itemdecl}
interrupt_token(bool initial_state) noexcept;
\end{itemdecl}
\begin{itemdescr}
  \pnum\effects Constructs a new \tcode{interrupt_token} object that can signal interrupts.

  \pnum\postconditions \tcode{valid() == true} and \tcode{is_interrupted() == initial_state}.
\end{itemdescr}

\indexlibrary{\idxcode{interrupt_token}!constructor}%
\begin{itemdecl}
interrupt_token(const interrupt_token& it) noexcept;
\end{itemdecl}
\begin{itemdescr}
  \pnum\effects If \tcode{it} is not valid, constructs an \tcode{interrupt_token} object
                that is not valid;
                otherwise, constructs an \tcode{interrupt_token}
                that shares the ownership of the interrupt signal with \tcode{it}.

  \pnum\postconditions \tcode{valid() == true}.
\end{itemdescr}

\indexlibrary{\idxcode{interrupt_token}!constructor}%
\begin{itemdecl}
interrupt_token(interrupt_token&& it) noexcept;
\end{itemdecl}
\begin{itemdescr}
  \pnum\effects Move constructs an object of type \tcode{interrupt_token} from \tcode{it}.

  \pnum\postconditions \tcode{*this} shall contain the old value of \tcode{it}.
                        \tcode{it.valid() == false}.
\end{itemdescr}

\rSec3[thread.thread.destr]{\tcode{thread} destructor}

\indexlibrary{\idxcode{interrupt_token}!destructor}%
\begin{itemdecl}
~interrupt_token();
\end{itemdecl}

\begin{itemdescr}
 \pnum\effects If \tcode{valid()} and this is the last owner of the interrupt signal,
                destructs the underlying interrupt signal.
\end{itemdescr}

%*****
\rSec3[interrupt_token.assign]{\tcode{interrupt_token} assign}

\indexlibrarymember{operator=}{interrupt_token}%
\begin{itemdecl}
interrupt_token& operator=(const interrupt_token& it) noexcept;
\end{itemdecl}
\begin{itemdescr}
  \pnum\effects Equivalent to: \tcode{interrupt_token(it).swap(*this);}

  \pnum\returns \tcode{*this}.
\end{itemdescr}

\indexlibrarymember{operator=}{interrupt_token}%
\begin{itemdecl}
interrupt_token& operator=(interrupt_token&& it) noexcept;
\end{itemdecl}
\begin{itemdescr}
  \pnum\effects Equivalent to: \tcode{interrupt_token(std::move(it)).swap(*this);}

  \pnum\returns \tcode{*this}.
\end{itemdescr}

%*****
\rSec3[interrupt_token.swap]{\tcode{interrupt_token} swap}

\indexlibrarymember{swap}{interrupt_token}%
\begin{itemdecl}
void swap(interrupt_token& it) noexcept;
\end{itemdecl}

\begin{itemdescr}
\pnum
\effects Swaps the state of \tcode{*this} and \tcode{it}.
\end{itemdescr}


%*****
\rSec3[interrupt_token.mem]{\tcode{interrupt_token} members}

\indexlibrarymember{valid}{interrupt_token}%
\begin{itemdecl}
bool valid() const noexcept;
\end{itemdecl}
\begin{itemdescr}
  \pnum\returns \tcode{true} if the interrupt token can be used to signal interrupts.
\end{itemdescr}

\indexlibrarymember{is_interrupted}{interrupt_token}%
\begin{itemdecl}
bool is_interrupted() const noexcept;
\end{itemdecl}
\begin{itemdescr}
  \pnum\returns \tcode{true} if initialized with \tcode{true}
                or initialized with \tcode{false} and \tcode{interrupt()} was called
                by one of the owners.
\end{itemdescr}

\indexlibrarymember{interrupt}{interrupt_token}%
\begin{itemdecl}
bool interrupt();
\end{itemdecl}
\begin{itemdescr}
  \pnum\requires \tcode{valid() == true}

  \pnum\effects \tcode{is_interrupted() == true}.

  \pnum\returns The value of \tcode{is_interrupted()} prior to the call.
\end{itemdescr}

\indexlibrarymember{throw_if_interrupted}{interrupt_token}%
\begin{itemdecl}
void throw_if_interrupted() const;
\end{itemdecl}
\begin{itemdescr}
  \pnum\effects Equivalent to:
\begin{codeblock}
if (is_interrupted())
  throw interrupted();
\end{codeblock}
\end{itemdescr}


%*****
\rSec3[interrupt_token.cmp]{\tcode{interrupt_token} comparisons}

\indexlibrarymember{operator==}{interrupt_token}%
\begin{itemdecl}
bool operator== (const interrupt_token& lhs, const interrupt_token& rhs);
\end{itemdecl}
\begin{itemdescr}
  \pnum\returns \tcode{!lhs.valid() \&\& !rhs.valid()} or
                whether \tcode{lhs} and \tcode{rhs} refer to the
                same interrupt token
                (copied or moved from the same initial object).
\end{itemdescr}

\indexlibrarymember{operator!=}{interrupt_token}%
\begin{itemdecl}
bool operator!= (const interrupt_token& lhs, const interrupt_token& rhs);
\end{itemdecl}
\begin{itemdescr}
  \pnum\returns \tcode{!(lhs==rhs)}.
\end{itemdescr}


\clearpage

%******************************************************************
\rSec1[thread.jthreads]{Joining Threads}

{\color{blue}
[{\itshape{}Editorial note:} {\color{diffcolor}This color signals differences to class \tcode{std::thread}.} ]
}


\pnum
\ref{thread.jthreads} describes components that can be used to create and manage threads
with the ability to signal interrupts to cooperatively cancel the running thread.

%**************************
\rSec2[thread.jthread.syn]{Header \tcode{<jthread>} synopsis}
\indexhdr{jthread}%

\begin{codeblock}
#include <interrupt_token>

namespace std {
  // \ref{thread.jthread.class} class \tcode{jthread}
  class jthread;

  void swap(jthread& x, jthread& y) noexcept;

  // \ref{thread.jthread.this} class \tcode{jthread}
  namespace this_thread {
    static bool is_interrupted() noexcept;
    static void throw_if_interrupted();
    static interrupt_token get_interrupt_token() noexcept;
    static interrupt_token exchange_interrupt_token(const interrupt_token&) noexcept;
  }
}
\end{codeblock}

%**************************
\rSec2[thread.jthread.class]{Class \tcode{jthread}}

\pnum
The class \tcode{jthread} provides a mechanism on top of class \tcode{thread} (\ref{thread.thread}),
which the additional ability to signal interrupts and
let the destructor \tcode{join()} if still \tcode{joinable()}.
The functionality is identical to class \tcode{thread}
except where otherwise specified.

\indexlibrary{\idxcode{jthread}}%
\begin{codeblock}
namespace std {
  class jthread {
  public:
    // types
    using id = thread::id;
    using native_handle_type = thread::native_handle_type;

    // construct/copy/destroy
    jthread() noexcept;
    template<class F, class... Args> explicit jthread(F&& f, Args&&... args);
    ~jthread();
    jthread(const jthread&) = delete;
    jthread(jthread&&) noexcept;
    jthread& operator=(const jthread&) = delete;
    jthread& operator=(jthread&&) noexcept;

    // members
    void swap(jthread&) noexcept;
    bool joinable() const noexcept;
    void join();
    void detach();
    id get_id() const noexcept;
    native_handle_type native_handle();     // see~\ref{thread.req.native}
\end{codeblock}
{\color{diffcolor}
\begin{codeblock}
    // interrupt token handling
    interrupt_token get_original_interrupt_token() const noexcept;
    bool interrupt() noexcept;
\end{codeblock}
}%\color{diffcolor}
\begin{codeblock}
    // static members
    static unsigned int hardware_concurrency() noexcept;
  
\end{codeblock}
{\color{diffcolor}
\begin{codeblock}
  private:
    interrupt_token itoken;                 // \expos
  };
\end{codeblock}
}%\color{diffcolor}
\begin{codeblock}
}
\end{codeblock}


\rSec3[thread.jthread.constr]{\tcode{jthread} constructors}

\indexlibrary{\idxcode{jthread}!constructor}%
\begin{itemdecl}
jthread() noexcept;
\end{itemdecl}
\begin{itemdescr}
  \pnum\effects Constructs a \tcode{jthread} object that does not represent a thread of execution.

  \pnum\postconditions \tcode{get_id() == id()}
        {\color{diffcolor} and \tcode{itoken.valid() == false}}.
\end{itemdescr}

\indexlibrary{\idxcode{jthread}!constructor}%
\begin{itemdecl}
template<class F, class... Args> explicit jthread(F&& f, Args&&... args);
\end{itemdecl}
\begin{itemdescr}
  \pnum
\requires\ \tcode{F} and each $\tcode{T}_i$ in \tcode{Args} shall satisfy the
\oldconcept{MoveConstructible} requirements.
\tcode{%
\placeholdernc{INVOKE}(\brk{}%
\placeholdernc{DECAY_COPY}(\brk{}%
std::forward<F>(f)),
\placeholdernc{DECAY_COPY}(\brk{}%
std::forward<Args>(\brk{}args))...)}\iref{func.require} shall be
a valid expression.

\pnum\remarks
This constructor shall not participate in overload resolution if \tcode{remove_cvref_t<F>}
is the same type as \tcode{std::jthread}.

\pnum\effects\ Constructs an object of type \tcode{jthread}. The new thread of execution executes
\tcode{%
\placeholdernc{INVOKE}(\brk{}%
\placeholdernc{DECAY_COPY}(\brk{}%
std::forward<F>(f)),
\placeholdernc{DECAY_COPY}(\brk{}%
std::forward<Args>(\brk{}args))...)} with the calls to
\tcode{\placeholder{DECAY_COPY}} being evaluated in the constructing thread. Any return value from this invocation
is ignored. \begin{note} This implies that any exceptions not thrown from the invocation of the copy
of \tcode{f} will be thrown in the constructing thread, not the new thread. \end{note} If the
invocation of
\tcode{%
\placeholdernc{INVOKE}(\brk{}%
\placeholdernc{DECAY_COPY}(\brk{}%
std::forward<F>(f)),
\placeholdernc{DECAY_COPY}(\brk{}%
std::forward<Args>(args))...)}
termi\-nates with an uncaught exception, \tcode{terminate} shall be called.

{\color{diffcolor}
An uncaught \tcode{interrupted} exception in the started thread of execution
will silently be ignored.
\begin{note} Thus, an uncaught exception thrown by \tcode{throw_if_interrupted()}
             will cause the started thread to end silently.
             \end{note}
}%\color{diffcolor}

\pnum\sync The completion of the invocation of the constructor
synchronizes with the beginning of the invocation of the copy of \tcode{f}.

{\color{diffcolor}
\pnum\postconditions \tcode{get_id() != id()}.
                     \tcode{itoken.valid() == true}.
                     \tcode{*this} represents the newly started thread.
                     In the started thread of execution \tcode{this_thread::thread_itoken}
                     is an \tcode{interrupt_token} equal to \tcode{itoken}. 
\begin{note} Note that the calling thread can signal an interrupt only once,
                because it can't replace this interrupt token.  \end{note}
}%\color{diffcolor}

\pnum\throws \tcode{system_error} if unable to start the new thread.

\pnum\errors
\begin{itemize}
\item \tcode{resource_unavailable_try_again} --- the system lacked the necessary
resources to create another thread, or the system-imposed limit on the number of
threads in a process would be exceeded.
\end{itemize}
\end{itemdescr}

\indexlibrary{\idxcode{jthread}!constructor}%
\begin{itemdecl}
jthread(jthread&& x) noexcept;
\end{itemdecl}

\begin{itemdescr}
\pnum
\effects Constructs an object of type \tcode{jthread} from \tcode{x}, and sets
\tcode{x} to a default constructed state.

\pnum
\postconditions \tcode{x.get_id() == id()} and \tcode{get_id()} returns the
value of \tcode{x.get_id()} prior to the start of construction.
{\color{diffcolor}
\tcode{itoken} yields the value of \tcode{x.itoken} prior to the start of construction
and \tcode{x.itoken.valid() == false}.
}%\color{diffcolor}

\end{itemdescr}

\rSec3[thread.jthread.destr]{\tcode{jthread} destructor}

\indexlibrary{\idxcode{jthread}!destructor}%
\begin{itemdecl}
~jthread();
\end{itemdecl}

{\color{diffcolor}
\begin{itemdescr}
\pnum
If \tcode{joinable()}, calls \tcode{interrupt()} and \tcode{join()}.
Otherwise, has no effects.
\begin{note} Operations on \tcode{*this} are not synchronized. \end{note}
\end{itemdescr}
}%\color{diffcolor}

\rSec3[thread.jthread.assign]{\tcode{jthread} assignment}

\indexlibrarymember{operator=}{jthread}%
\begin{itemdecl}
jthread& operator=(jthread&& x) noexcept;
\end{itemdecl}

\begin{itemdescr}
\pnum
\effects If \tcode{joinable()}, calls \tcode{interrupt()} and \tcode{join()}.
Assigns the
state of \tcode{x} to \tcode{*this} and sets \tcode{x} to a default constructed state.

{\color{diffcolor}
\pnum
\postconditions \tcode{x.get_id() == id()} and \tcode{get_id()} returns the value of
\tcode{x.get_id()} prior to the assignment.
\tcode{itoken} yields the value of \tcode{x.itoken} prior to the assignment
and \tcode{x.itoken.valid() == false}.
}%\color{diffcolor}

\pnum
\returns \tcode{*this}.
\end{itemdescr}


\rSec3[thread.jthread.interrupt]{\tcode{jthread} interrupt members}

{\color{diffcolor}
\indexlibrarymember{get_original_interrupt_token}{jthread}%
\begin{itemdecl}
interrupt_token get_original_interrupt_token() const noexcept
\end{itemdecl}
\begin{itemdescr}
  \pnum\effects Equivalent to: \tcode{return itoken;}
\end{itemdescr}

\indexlibrarymember{interrupt}{jthread}%
\begin{itemdecl}
bool interrupt() noexcept;
\end{itemdecl}
\begin{itemdescr}
  \pnum\effects Equivalent to: \tcode{return itoken.interrupt();}
\end{itemdescr}
}%\color{diffcolor}

%\rSec3[thread.jthread.static]{\tcode{jthread} static members}
%
%\indexlibrarymember{hardware_concurrency}{thread}%
%\begin{itemdecl}
%unsigned hardware_concurrency() noexcept;
%\end{itemdecl}
%\begin{itemdescr}
%  \pnum\effects Equivalent to: \tcode{return thread::hardware_concurrency()}.
%\end{itemdescr}
%
%\rSec3[thread.jthread.algorithm]{\tcode{jthread} specialized algorithms}
%
%\indexlibrarymember{swap}{jthread}%
%\begin{itemdecl}
%void swap(jthread& x, jthread& y) noexcept;
%\end{itemdecl}
%\begin{itemdescr}
%  \pnum\effects As if by \tcode{x.swap(y)}.
%\end{itemdescr}

%**************************
\rSec2[thread.jthread.this]{Namespace \tcode{this_thread} Interrupt Handling}

To be able to deal with signaled interrupt
\tcode{this_thread} provides the following access to an \tcode{interrupt_token} (\ref{interrupt_token}):

\begin{codeblock}
namespace std::this_thread {
  interrupt_token thread_itoken;                 // \expos
  static interrupt_token get_interrupt_token() noexcept;
  static bool is_interrupted() noexcept;
  static void throw_if_interrupted();
  static interrupt_token exchange_interrupt_token(const interrupt_token&) noexcept;
}
\end{codeblock}

For any thread of execution, \tcode{thread_itoken} is default initialized unless
the thread was started with a constructor of class \tcode{jthread} (\ref{thread.jthread}).


\indexlibrarymember{get_interrupt_token}{this_thread}%
\begin{itemdecl}
interrupt_token get_interrupt_token() noexcept;
\end{itemdecl}
\begin{itemdescr}
  \pnum\returns \tcode{this_thread::thread_itoken}.
\end{itemdescr}

\indexlibrarymember{is_interrupted}{this_thread}%
\begin{itemdecl}
bool is_interrupted() noexcept;
\end{itemdecl}
\begin{itemdescr}
  \pnum\returns \tcode{this_thread::get_interrupt_token().is_interrupted()}.
\end{itemdescr}

\indexlibrarymember{throw_if_interrupted}{this_thread}%
\begin{itemdecl}
void throw_if_interrupted();
\end{itemdecl}
\begin{itemdescr}
  \pnum\effects Equivalent to: \tcode{this_thread::get_interrupt_token().throw_if_interrupted();}
\end{itemdescr}

\indexlibrarymember{exchange_interrupt_token}{this_thread}%
\begin{itemdecl}
interrupt_token exchange_interrupt_token(const interrupt_token& it) noexcept;
\end{itemdecl}
\begin{itemdescr}
  \pnum\effects Equivalent to: \tcode{this_thread::itoken = it;}

  \pnum\returns \tcode{this_thread::thread_itoken} prior to the exchange.

  \begin{note} With this exchange \tcode{this_thread::get_interrupt_token()}
               will no longer signal interrupts from the calling threads
               until the returned token is restored.
               \end{note}
\end{itemdescr}

}%\color{insertcolor}

\clearpage

%******************************************************************
\rSec1[thread.condition]{Condition variables}

%\rSec2[condition_variable.syn]{Header \tcode{<condition_variable>} synopsis}
%\indexhdr{condition_variable}%
%
%%{NEW: add interrupted:}
%
%\begin{codeblock}
%namespace std {
%  enum class cv_status { no_timeout, timeout,
%\end{codeblock}
%{\color{insertcolor}
%\begin{codeblock}
%                         interrupted
%\end{codeblock}
%}
%\begin{codeblock}
%                       };
%}
%\end{codeblock}


\rSec2[thread.condition.condvar]{Class \tcode{condition_variable}}

\indexlibrary{\idxcode{condition_variable}}%
\begin{codeblock}
namespace std {
  class condition_variable {
  public:
    condition_variable();
    ~condition_variable();

    condition_variable(const condition_variable&) = delete;
    condition_variable& operator=(const condition_variable&) = delete;

    void notify_one() noexcept;
    void notify_all() noexcept;

    void wait(unique_lock<mutex>& lock);
    template<class Predicate>
      void wait(unique_lock<mutex>& lock, Predicate pred);

    template<class Clock, class Duration>
      cv_status wait_until(unique_lock<mutex>& lock,
                           const chrono::time_point<Clock, Duration>& abs_time);
    template<class Clock, class Duration, class Predicate>
      bool wait_until(unique_lock<mutex>& lock,
                      const chrono::time_point<Clock, Duration>& abs_time,
                      Predicate pred);

    template<class Rep, class Period>
      cv_status wait_for(unique_lock<mutex>& lock,
                         const chrono::duration<Rep, Period>& rel_time);
    template<class Rep, class Period, class Predicate>
      bool wait_for(unique_lock<mutex>& lock,
                    const chrono::duration<Rep, Period>& rel_time,
                    Predicate pred);
\end{codeblock}
{\color{insertcolor}
\begin{codeblock}
    // \ref{thread.condition.interrupted} dealing with interrupts:

    // throw \tcode{std::interrupted} if \tcode{this_thread::is_interrupted()}:
    template<class Predicate>
      void iwait(unique_lock<mutex>& lock, Predicate pred);

    template<class Clock, class Duration, class Predicate>
      bool iwait_until(unique_lock<mutex>& lock,
                       const chrono::time_point<Clock, Duration>& abs_time,
                       Predicate pred);
    template<class Rep, class Period, class Predicate>
      bool iwait_for(unique_lock<mutex>& lock,
                     const chrono::duration<Rep, Period>& rel_time,
                     Predicate pred);

    // return \tcode{false} if \tcode{itoken.is_interrupted()}:
    template <class Predicate>
      bool wait_until(unique_lock<mutex>& lock,
                      Predicate pred,
                      interrupt_token itoken);
    template <class Clock, class Duration, class Predicate>
      bool wait_until(unique_lock<mutex>& lock,
                      const chrono::time_point<Clock, Duration>& abs_time
                      Predicate pred,
                      interrupt_token itoken);
    template <class Rep, class Period, class Predicate>
      bool wait_for(unique_lock<mutex>& lock,
                    const chrono::duration<Rep, Period>& rel_time,
                    Predicate pred,
                    interrupt_token itoken);
\end{codeblock}
}
\begin{codeblock}
    using native_handle_type = @\impdefnc@;          // see~\ref{thread.req.native}
    native_handle_type native_handle();                         // see~\ref{thread.req.native}
  };
}
\end{codeblock}


\pnum
The class \tcode{condition_variable} shall be a standard-layout class\iref{class.prop}.

\indexlibrary{\idxcode{condition_variable}!constructor}%
\begin{itemdecl}
condition_variable();
\end{itemdecl}

\begin{itemdescr}
\pnum
\effects Constructs an object of type \tcode{condition_variable}.

\pnum
\throws \tcode{system_error} when an exception is required\iref{thread.req.exception}.

\pnum
\errors
\begin{itemize}
\item \tcode{resource_unavailable_try_again} --- if some non-memory resource
limitation prevents initialization.
\end{itemize}
\end{itemdescr}

\indexlibrary{\idxcode{condition_variable}!destructor}%
\begin{itemdecl}
~condition_variable();
\end{itemdecl}

\begin{itemdescr}
\pnum
\requires There shall be no thread blocked on \tcode{*this}. \begin{note} That is, all
threads shall have been notified; they may subsequently block on the lock specified in the
wait.
This relaxes the usual rules, which would have required all wait calls to happen before
destruction. Only the notification to unblock the wait needs to happen before destruction.
The user should take care to ensure that no threads wait on \tcode{*this} once the destructor has
been started, especially when the waiting threads are calling the wait functions in a loop or
using the overloads of \tcode{wait}, \tcode{wait_for}, or \tcode{wait_until} that take a predicate.
\end{note}

\pnum\effects Destroys the object.
\end{itemdescr}

\indexlibrarymember{notify_one}{condition_variable}%
\begin{itemdecl}
void notify_one() noexcept;
\end{itemdecl}

\begin{itemdescr}
\pnum\effects If any threads are blocked waiting for \tcode{*this}, unblocks one of those threads.
\end{itemdescr}

\indexlibrarymember{notify_all}{condition_variable}%
\begin{itemdecl}
void notify_all() noexcept;
\end{itemdecl}


%*** wait functions:

\begin{itemdescr}
\pnum\effects Unblocks all threads that are blocked waiting for \tcode{*this}.
\end{itemdescr}

\indexlibrarymember{wait}{condition_variable}%
\begin{itemdecl}
void wait(unique_lock<mutex>& lock);
\end{itemdecl}
\begin{itemdescr}
 \pnum \requires \tcode{lock.owns_lock()} is \tcode{true} and \tcode{lock.mutex()}
        is locked by the calling thread, and either
        \begin{itemize}
         \item no other thread is waiting on this \tcode{condition_variable} object or
         \item \tcode{lock.mutex()} returns the same value for each of the \tcode{lock}
                arguments supplied by all concurrently waiting (via \tcode{wait},
                \tcode{wait_for}, \tcode{wait_until}
                {\color{insertcolor}, \tcode{iwait}, \tcode{iwait_for}, or \tcode{iwait_until}}) threads.
        \end{itemize}

 \pnum\effects
 \begin{itemize}
        \item Atomically calls \tcode{lock.unlock()} and blocks on \tcode{*this}.
        \item When unblocked, calls \tcode{lock.lock()} (possibly blocking on the lock), then returns.
        \item The function will unblock when signaled by a call to \tcode{notify_one()}
              or a call to \tcode{notify_all()}, or spuriously.
 \end{itemize}

 \pnum \remarks
        If the function fails to meet the postcondition, \tcode{terminate()}
        shall be called\iref{except.terminate}.
        \begin{note} This can happen if the re-locking of the mutex throws an exception. \end{note}
 
 \pnum \postconditions \tcode{lock.owns_lock()} is \tcode{true} and \tcode{lock.mutex()}
        is locked by the calling thread.

 \pnum\throws Nothing.
\end{itemdescr}


\indexlibrarymember{wait}{condition_variable}%
\begin{itemdecl}
template<class Predicate>
  void wait(unique_lock<mutex>& lock, Predicate pred);
\end{itemdecl}
\begin{itemdescr}
 \pnum \requires \tcode{lock.owns_lock()} is \tcode{true} and \tcode{lock.mutex()} is
        locked by the calling thread, and either

        \begin{itemize}
         \item no other thread is waiting on this \tcode{condition_variable} object or
         \item \tcode{lock.mutex()} returns the same value for each of the \tcode{lock}
                arguments supplied by all concurrently waiting (via \tcode{wait},
                \tcode{wait_for}, \tcode{wait_until}
                {\color{insertcolor}, \tcode{iwait}, \tcode{iwait_for}, or \tcode{iwait_until}}) threads.
        \end{itemize}

 \pnum \effects Equivalent to:
\begin{codeblock}
while (!pred())
  wait(lock);
\end{codeblock}

 \pnum \remarks
        If the function fails to meet the postcondition, \tcode{terminate()}
        shall be called\iref{except.terminate}.
        \begin{note} This can happen if the re-locking of the mutex throws an exception. \end{note}

 \pnum \postconditions \tcode{lock.owns_lock()} is \tcode{true} and \tcode{lock.mutex()}
        is locked by the calling thread.

 \pnum \throws Any exception thrown by \tcode{pred}.
\end{itemdescr}

%**** wait_until (no pred)

\indexlibrarymember{wait_until}{condition_variable}%
\begin{itemdecl}
template<class Clock, class Duration>
  cv_status wait_until(unique_lock<mutex>& lock,
                       const chrono::time_point<Clock, Duration>& abs_time);
\end{itemdecl}
\begin{itemdescr}
 \pnum \requires \tcode{lock.owns_lock()} is \tcode{true} and \tcode{lock.mutex()}
        is locked by the calling thread, and either
        \begin{itemize}
         \item no other thread is waiting on this \tcode{condition_variable} object or
         \item \tcode{lock.mutex()} returns the same value for each of the \tcode{lock}
                arguments supplied by all concurrently waiting (via \tcode{wait},
                \tcode{wait_for}, \tcode{wait_until}
                {\color{insertcolor}, \tcode{iwait}, \tcode{iwait_for}, or \tcode{iwait_until}}) threads.
        \end{itemize}

 \pnum \effects
        \begin{itemize}
         \item Atomically calls \tcode{lock.unlock()} and blocks on \tcode{*this}.

         \item When unblocked, calls \tcode{lock.lock()} (possibly blocking on the lock), then returns.

         \item The function will unblock when signaled by a call to \tcode{notify_one()}, a call to \tcode{notify_all()},
                expiration of the absolute timeout\iref{thread.req.timing} specified by \tcode{abs_time},
                or spuriously.

         \item If the function exits via an exception, \tcode{lock.lock()} shall be called prior to exiting the function.
        \end{itemize}

 \pnum \remarks
        If the function fails to meet the postcondition, \tcode{terminate()}
        shall be called\iref{except.terminate}.
        \begin{note} This can happen if the re-locking of the mutex throws an exception. \end{note}

 \pnum \postconditions \tcode{lock.owns_lock()} is \tcode{true} and \tcode{lock.mutex()}
        is locked by the calling thread.

 \pnum \returns \tcode{cv_status::timeout} if
        the absolute timeout\iref{thread.req.timing} specified by \tcode{abs_time} expired,
        otherwise \tcode{cv_status::no_timeout}.

 \pnum\throws Timeout-related exceptions\iref{thread.req.timing}.
\end{itemdescr}

%**** wait_for (no pred)

\indexlibrarymember{wait_for}{condition_variable}%
\begin{itemdecl}
template<class Rep, class Period>
  cv_status wait_for(unique_lock<mutex>& lock,
                     const chrono::duration<Rep, Period>& rel_time);
\end{itemdecl}
\begin{itemdescr}
 \pnum \requires \tcode{lock.owns_lock()} is \tcode{true} and \tcode{lock.mutex()}
        is locked by the calling thread, and either
        \begin{itemize}
         \item no other thread is waiting on this \tcode{condition_variable} object or
         \item \tcode{lock.mutex()} returns the same value for each of the \tcode{lock} arguments
                supplied by all concurrently waiting (via \tcode{wait},
                \tcode{wait_for}, \tcode{wait_until}
                {\color{insertcolor}, \tcode{iwait}, \tcode{iwait_for}, or \tcode{iwait_until}}) threads.
        \end{itemize}

 \pnum \effects Equivalent to:
\begin{codeblock}
return wait_until(lock, chrono::steady_clock::now() + rel_time);
\end{codeblock}

 \pnum \returns \tcode{cv_status::timeout} if
                the relative timeout\iref{thread.req.timing} specified by \tcode{rel_time} expired,
                otherwise \tcode{cv_status::no_timeout}.

 \pnum \remarks
                If the function fails to meet the postcondition, \tcode{terminate()}
                shall be called\iref{except.terminate}.
                \begin{note} This can happen if the re-locking of the mutex throws an exception. \end{note}

 \pnum \postconditions \tcode{lock.owns_lock()} is \tcode{true} and \tcode{lock.mutex()}
        is locked by the calling thread.

 \pnum \throws Timeout-related exceptions\iref{thread.req.timing}.
\end{itemdescr}

%**** wait_until (with pred)

\indexlibrarymember{wait_until}{condition_variable}%
\begin{itemdecl}
template<class Clock, class Duration, class Predicate>
  bool wait_until(unique_lock<mutex>& lock,
                  const chrono::time_point<Clock, Duration>& abs_time,
                  Predicate pred);
\end{itemdecl}
\begin{itemdescr}
 \pnum \requires \tcode{lock.owns_lock()} is \tcode{true} and \tcode{lock.mutex()} is
                locked by the calling thread, and either

        \begin{itemize}
        \item no other thread is waiting on this \tcode{condition_variable} object or
        \item \tcode{lock.mutex()} returns the same value for each of the \tcode{lock}
                arguments supplied by all concurrently waiting (via \tcode{wait},
                \tcode{wait_for}, \tcode{wait_until}
                {\color{insertcolor}, \tcode{iwait}, \tcode{iwait_for}, or \tcode{iwait_until}}) threads.
        \end{itemize}

 \pnum \effects Equivalent to:
\begin{codeblock}
while (!pred())
  if (wait_until(lock, abs_time) == cv_status::timeout)
    return pred();
return true;
\end{codeblock}

 \pnum \remarks
        If the function fails to meet the postcondition, \tcode{terminate()}
        shall be called\iref{except.terminate}.
        \begin{note} This can happen if the re-locking of the mutex throws an exception. \end{note}

 \pnum \postconditions \tcode{lock.owns_lock()} is \tcode{true} and \tcode{lock.mutex()}
        is locked by the calling thread.

 \pnum \begin{note} The returned value indicates whether the predicate evaluated to
        \tcode{true} regardless of whether the timeout was triggered. \end{note}

 \pnum \throws Timeout-related exceptions\iref{thread.req.timing} or any exception thrown by \tcode{pred}.
\end{itemdescr}


%**** wait_for (with pred)

\indexlibrarymember{wait_for}{condition_variable}%
\begin{itemdecl}
template<class Rep, class Period, class Predicate>
  bool wait_for(unique_lock<mutex>& lock,
                const chrono::duration<Rep, Period>& rel_time,
                Predicate pred);
\end{itemdecl}
\begin{itemdescr}
 \pnum \requires \tcode{lock.owns_lock()} is \tcode{true} and \tcode{lock.mutex()}
        is locked by the calling thread, and either
        \begin{itemize}
        \item no other thread is waiting on this \tcode{condition_variable} object or
        \item \tcode{lock.mutex()} returns the same value for each of the \tcode{lock} arguments
                supplied by all concurrently waiting (via \tcode{wait},
                \tcode{wait_for}, \tcode{wait_until}
                {\color{insertcolor}, \tcode{iwait}, \tcode{iwait_for}, or \tcode{iwait_until}}) threads.
        \end{itemize}
 
 \pnum \effects Equivalent to:
\begin{codeblock}
return wait_until(lock, chrono::steady_clock::now() + rel_time, std::move(pred));
\end{codeblock}
 
 \pnum \begin{note} There is no blocking if \tcode{pred()} is initially \tcode{true}, even if the
        timeout has already expired. \end{note}
 
 \pnum \remarks
        If the function fails to meet the postcondition, \tcode{terminate()}
        shall be called\iref{except.terminate}.
        \begin{note} This can happen if the re-locking of the mutex throws an exception. \end{note}
 
 \pnum \postconditions \tcode{lock.owns_lock()} is \tcode{true} and \tcode{lock.mutex()}
        is locked by the calling thread.
 
 \pnum \begin{note} The returned value indicates whether the predicate evaluates to \tcode{true}
        regardless of whether the timeout was triggered. \end{note}
 
 \pnum \throws Timeout-related exceptions\iref{thread.req.timing} or any exception thrown by \tcode{pred}.
\end{itemdescr}




%**** wait







%**********************************************************
%\begin{codeblock}
%******************
%* NEW:
%******************
%\end{codeblock}
{\color{insertcolor}

%// \ref{thread.condition.interupted} dealing with interrupts:
\rSec3[thread.condition.interrupted]{\tcode{interrupt_token} handling}

The following functions respect the state of the \tcode{interrupt_token}
passed as argument or returned by \tcode{this_thread::get_interrupt_token()}.
The functions starting with \tcode{i}
(\tcode{iwait()}, \tcode{iwait_until()}, \tcode{iwait_for()}
throw \tcode{std::interrupted} on a signaled interrupt.
The other functions listed here
(\tcode{wait_until()}, \tcode{wait_for()}
return \tcode{false} on a signaled interrupt
(if the predicate evaluates to \tcode{false}). 

% Hans Boehm:
%We then also need some statement for the waiting functions that if they return with an interrupted status, their synchronization behavior is as though they called is_interrupted().

{\color{blue}
[{\itshape{}Editorial note:} {\color{diffcolor}This color signals differences to the corresponding \tcode{wait...()} function.} ]
}

\indexlibrarymember{iwait}{condition_variable}%
\begin{itemdecl}
template<class Predicate>
  void iwait(unique_lock<mutex>& lock, Predicate pred);
\end{itemdecl}
\begin{itemdescr}
 \pnum \requires \tcode{lock.owns_lock()} is \tcode{true} and \tcode{lock.mutex()} is
        locked by the calling thread, and either

        \begin{itemize}
         \item no other thread is waiting on this \tcode{condition_variable} object or
         \item \tcode{lock.mutex()} returns the same value for each of the \tcode{lock}
                arguments supplied by all concurrently waiting (via \tcode{wait},
                \tcode{wait_for}, \tcode{wait_until}
                {\color{insertcolor}, \tcode{iwait}, \tcode{iwait_for}, or \tcode{iwait_until}}) threads.
        \end{itemize}

{\color{diffcolor}
  \pnum\effects Registers \tcode{*this} to get notified if an interrupt is signaled on \tcode{this_thread::get_interrupt_token()}
                during this call and then equivalent to:
\begin{codeblock}
while(!pred()) {
  this_thread::throw_if_interrupted();
  cv.wait(lock, [&pred] {
                  return pred() || this_thread::is_interrupted();
                });
}
\end{codeblock}
}%diffcolor
  
 \pnum \remarks
        If the function fails to meet the postcondition, \tcode{terminate()}
        shall be called\iref{except.terminate}.
        \begin{note} This can happen if the re-locking of the mutex throws an exception. \end{note}

 \pnum \postconditions \tcode{lock.owns_lock()} is \tcode{true} and \tcode{lock.mutex()}
        is locked by the calling thread.

 \pnum \throws Any exception thrown by \tcode{pred}
                {\color{diffcolor}or exception \tcode{interrupted} if
                 \tcode{this_thread::is_interrupted()}}.

{\color{diffcolor}
 \pnum\sync If the function returns with an interrupted status, 
                their synchronization behavior is as though it called \tcode{is_interrupted()}.
}%diffcolor
\end{itemdescr}



\indexlibrarymember{iwait_until}{condition_variable}%
\begin{itemdecl}
template<class Clock, class Duration, class Predicate>
  bool iwait_until(unique_lock<mutex>& lock,
                   const chrono::time_point<Clock, Duration>& abs_time,
                   Predicate pred);
\end{itemdecl}
\begin{itemdescr}
 \pnum \requires \tcode{lock.owns_lock()} is \tcode{true} and \tcode{lock.mutex()} is
                locked by the calling thread, and either

        \begin{itemize}
        \item no other thread is waiting on this \tcode{condition_variable} object or
        \item \tcode{lock.mutex()} returns the same value for each of the \tcode{lock}
                arguments supplied by all concurrently waiting (via \tcode{wait},
                \tcode{wait_for}, \tcode{wait_until}
                {\color{insertcolor}, \tcode{iwait}, \tcode{iwait_for}, or \tcode{iwait_until}}) threads.
        \end{itemize}

{\color{diffcolor}
  \pnum\effects Registers \tcode{*this} to get notified if an interrupt is signaled on \tcode{this_thread::get_interrupt_token()}
                during this call and then equivalent to:
\begin{codeblock}
while(!pred() && Clock::now() < abs_time) {
  this_thread::throw_if_interrupted();
  cv.wait_until(lock, abs_time,
                [&pred] {
                  return pred() || this_thread::is_interrupted();
                });
}
return pred();
\end{codeblock}
}%diffcolor

 \pnum \remarks
        If the function fails to meet the postcondition, \tcode{terminate()}
        shall be called\iref{except.terminate}.
        \begin{note} This can happen if the re-locking of the mutex throws an exception. \end{note}

 \pnum \postconditions \tcode{lock.owns_lock()} is \tcode{true} and \tcode{lock.mutex()}
        is locked by the calling thread.

 \pnum \begin{note} The returned value indicates whether the predicate evaluated to
        \tcode{true} regardless of whether the timeout was triggered
        {\color{diffcolor} or an interrupt was signaled}. \end{note}

 \pnum \throws Timeout-related exceptions\iref{thread.req.timing} or any exception thrown by \tcode{pred}
                {\color{diffcolor}or exception \tcode{interrupted} if
                 \tcode{this_thread::is_interrupted()}}.

{\color{diffcolor}
 \pnum\sync If the function returns with an interrupted status, 
                their synchronization behavior is as though it called \tcode{is_interrupted()}.
}%diffcolor
\end{itemdescr}



\indexlibrarymember{iwait_for}{condition_variable}%
\begin{itemdecl}
template<class Rep, class Period, class Predicate>
  bool iwait_for(unique_lock<mutex>& lock,
                 const chrono::duration<Rep, Period>& rel_time,
                 Predicate pred);
\end{itemdecl}
\begin{itemdescr}
 \pnum \requires \tcode{lock.owns_lock()} is \tcode{true} and \tcode{lock.mutex()} is
                locked by the calling thread, and either

        \begin{itemize}
        \item no other thread is waiting on this \tcode{condition_variable} object or
        \item \tcode{lock.mutex()} returns the same value for each of the \tcode{lock}
                arguments supplied by all concurrently waiting (via \tcode{wait},
                \tcode{wait_for}, \tcode{wait_until}
                {\color{insertcolor}, \tcode{iwait}, \tcode{iwait_for}, or \tcode{iwait_until}}) threads.
        \end{itemize}

{\color{diffcolor}
 \pnum \effects Equivalent to:
\begin{codeblock}
this_thread::throw_if_interrupted();
return iwait_until(lock,
                   std::chrono::steady_clock::now() + rel_time,
                   std::move(pred));
\end{codeblock}
}%diffcolor

 \pnum \remarks
        If the function fails to meet the postcondition, \tcode{terminate()}
        shall be called\iref{except.terminate}.
        \begin{note} This can happen if the re-locking of the mutex throws an exception. \end{note}

 \pnum \postconditions \tcode{lock.owns_lock()} is \tcode{true} and \tcode{lock.mutex()}
        is locked by the calling thread.

 \pnum \begin{note} The returned value indicates whether the predicate evaluated to
        \tcode{true} regardless of whether the timeout was triggered
        {\color{diffcolor} or an interrupt was signaled}. \end{note}

 \pnum \throws Timeout-related exceptions\iref{thread.req.timing} or any exception thrown by \tcode{pred}
                {\color{diffcolor}or exception \tcode{interrupted} if
                 \tcode{this_thread::is_interrupted()}}.

{\color{diffcolor}
 \pnum\sync If the function returns with an interrupted status, 
                their synchronization behavior is as though it called \tcode{is_interrupted()}.
}%diffcolor
\end{itemdescr}


%**** untimed wait_until (with pred)
% return false if itoken.is_interrupted():

{\color{blue}
[{\itshape{}Editorial note:} {\color{diffcolor}This color signals differences to the corresponding \tcode{wait()} function.} ]
}

\begin{itemdecl}
template <class Predicate>
  bool wait_until(unique_lock<mutex>& lock,
                  Predicate pred,
                  interrupt_token itoken);
\end{itemdecl}
\begin{itemdescr}
 \pnum \requires \tcode{lock.owns_lock()} is \tcode{true} and \tcode{lock.mutex()} is
        locked by the calling thread, and either
        \begin{itemize}
         \item no other thread is waiting on this \tcode{condition_variable} object or
         \item \tcode{lock.mutex()} returns the same value for each of the \tcode{lock}
                arguments supplied by all concurrently waiting (via \tcode{wait},
                \tcode{wait_for}, \tcode{wait_until}
                {\color{insertcolor}, \tcode{iwait}, \tcode{iwait_for}, or \tcode{iwait_until}}) threads.
        \end{itemize}

{\color{diffcolor}
 \pnum\effects Registers \tcode{*this} to get notified if an interrupt is signaled on \tcode{itoken}
                during this call and then equivalent to:
\begin{codeblock}
while(!pred() && !itoken.is_interrupted()) {
  cv.wait(lock, [&pred, &itoken] {
                  return pred() || itoken.is_interrupted();
                });
}
return pred();
\end{codeblock}
}%diffcolor

 \pnum \begin{note} The returned value indicates whether the predicate evaluated to
        \tcode{true} regardless of whether the timeout was triggered
        {\color{diffcolor} or an interrupt was signaled}. \end{note}

 \pnum \postconditions \tcode{lock.owns_lock()} is \tcode{true} and \tcode{lock.mutex()}
        is locked by the calling thread.

 \pnum \remarks
        If the function fails to meet the postcondition, \tcode{terminate()}
        shall be called\iref{except.terminate}.
        \begin{note} This can happen if the re-locking of the mutex throws an exception. \end{note}

 \pnum \throws Any exception thrown by \tcode{pred}.

{\color{diffcolor}
 \pnum\sync If the function returns with an interrupted status, 
                their synchronization behavior is as though it called \tcode{is_interrupted()}.
}%diffcolor
\end{itemdescr}


%**** wait_until (with pred)
% return false if itoken.is_interrupted():

{\color{blue}
[{\itshape{}Editorial note:} {\color{diffcolor}This color signals differences to the corresponding \tcode{wait_until()} function without interrupt token.} ]
}

\begin{itemdecl}
template <class Clock, class Duration, class Predicate>
  bool wait_until(unique_lock<mutex>& lock,
                  const chrono::time_point<Clock, Duration>& abs_time
                  Predicate pred,
                  interrupt_token itoken);
\end{itemdecl}
\begin{itemdescr}
 \pnum \requires \tcode{lock.owns_lock()} is \tcode{true} and \tcode{lock.mutex()} is
        locked by the calling thread, and either
        \begin{itemize}
         \item no other thread is waiting on this \tcode{condition_variable} object or
         \item \tcode{lock.mutex()} returns the same value for each of the \tcode{lock}
                arguments supplied by all concurrently waiting (via \tcode{wait},
                \tcode{wait_for}, \tcode{wait_until}
                {\color{insertcolor}, \tcode{iwait}, \tcode{iwait_for}, or \tcode{iwait_until}}) threads.
        \end{itemize}

{\color{diffcolor}
 \pnum\effects Registers \tcode{*this} to get notified if an interrupt is signaled on \tcode{itoken}
                during this call and then equivalent to:
\begin{codeblock}
while(!pred() && !itoken.is_interrupted() && Clock::now() < abs_time) {
  cv.wait_until(lock,
                abs_time,
                [&pred, &itoken] {
                  return pred() || itoken.is_interrupted();
                });
}
return pred();
\end{codeblock}
}%diffcolor

 \pnum \begin{note} The returned value indicates whether the predicate evaluated to
        \tcode{true} regardless of whether the timeout was triggered
        {\color{diffcolor} or an interrupt was signaled}. \end{note}

 \pnum \postconditions \tcode{lock.owns_lock()} is \tcode{true} and \tcode{lock.mutex()}
        is locked by the calling thread.

 \pnum \remarks
        If the function fails to meet the postcondition, \tcode{terminate()}
        shall be called\iref{except.terminate}.
        \begin{note} This can happen if the re-locking of the mutex throws an exception. \end{note}

 \pnum \throws Timeout-related exceptions\iref{thread.req.timing} or any exception thrown by \tcode{pred}.

{\color{diffcolor}
 \pnum\sync If the function returns with an interrupted status, 
                their synchronization behavior is as though it called \tcode{is_interrupted()}.
}%diffcolor
\end{itemdescr}


%**** wait_for (with pred)
% return false if itoken.is_interrupted():

{\color{blue}
[{\itshape{}Editorial note:} {\color{diffcolor}This color signals differences to the corresponding \tcode{wait_for()} function without interrupt token.} ]
}

\begin{itemdecl}
template <class Rep, class Period, class Predicate>
  bool wait_for(unique_lock<mutex>& lock,
                const chrono::duration<Rep, Period>& rel_time,
                Predicate pred,
                interrupt_token itoken);
\end{itemdecl}
\begin{itemdescr}
 \pnum \requires \tcode{lock.owns_lock()} is \tcode{true} and \tcode{lock.mutex()}
        is locked by the calling thread, and either
        \begin{itemize}
         \item no other thread is waiting on this \tcode{condition_variable} object or
         \item \tcode{lock.mutex()} returns the same value for each of the \tcode{lock} arguments
                supplied by all concurrently waiting (via \tcode{wait},
                \tcode{wait_for}, \tcode{wait_until}
                {\color{insertcolor}, \tcode{iwait}, \tcode{iwait_for}, or \tcode{iwait_until}}) threads.
        \end{itemize}

 \pnum \effects Equivalent to:
\begin{codeblock}
return wait_until(lock, chrono::steady_clock::now() + rel_time, std::move(pred),
\end{codeblock}
{\color{diffcolor}
\begin{codeblock}
                  std::move(itoken));
\end{codeblock}
}

 \pnum \begin{note} The returned value indicates whether the predicate evaluates to \tcode{true}
        regardless of whether the timeout was triggered 
        {\color{diffcolor} or an interrupt was signaled}. \end{note}

 \pnum \begin{note} There is no blocking if \tcode{pred()} is initially \tcode{true}, even if the
        timeout has already expired. \end{note}

 \pnum \postconditions \tcode{lock.owns_lock()} is \tcode{true} and \tcode{lock.mutex()}
        is locked by the calling thread.

 \pnum \remarks
        If the function fails to meet the postcondition, \tcode{terminate()}
        shall be called\iref{except.terminate}.
        \begin{note} This can happen if the re-locking of the mutex throws an exception. \end{note}
 
 \pnum \throws Timeout-related
        exceptions\iref{thread.req.timing} or any exception thrown by \tcode{pred}
        {\color{diffcolor} or \tcode{interrupted} if an interrupt was signaled}.

{\color{diffcolor}
 \pnum\sync If the function returns with an interrupted status, 
                their synchronization behavior is as though it called \tcode{is_interrupted()}.
}%diffcolor
\end{itemdescr}

}%\color{insertcolor}

