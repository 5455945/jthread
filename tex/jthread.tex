%!TEX root = std.tex
\setcounter{chapter}{29}
\rSec0[thread]{Thread support library}

%******************************************************************
\rSec1[jthread.general]{General}

\pnum
The following subclauses describe components to create and manage
threads\iref{intro.multithread}, perform mutual exclusion, and communicate conditions
and values
between threads, as summarized in \tref{thread.lib.summary}.

\begin{libsumtab}{Thread support library summary}{tab:thread.lib.summary}
\ref{thread.req}      & Requirements          &                               \\ \rowsep
\ref{thread.threads}  & Threads               & \tcode{<thread>}              \\ \rowsep
\color{insertcolor}
\ref{thread.stop_token} &
        \color{insertcolor} Stop Tokens       &
                \color{insertcolor} \tcode{<stop_token>}              \\ \rowsep
\color{insertcolor}
\ref{thread.jthreads} &
        \color{insertcolor} Joining Threads       &
                \color{insertcolor} \tcode{<jthread>}              \\ \rowsep
\ref{thread.mutex}    & Mutual exclusion      & \tcode{<mutex>}               \\
                      &                       & \tcode{<shared_mutex>}        \\ \rowsep
\ref{thread.condition}& Condition variables   & \tcode{<condition_variable>}  \\ \rowsep
\ref{futures}         & Futures               & \tcode{<future>}              \\
\end{libsumtab}

%******************************************************************
\rSec1[thread.req]{Requirements}

...

%******************************************************************
\rSec1[thread.threads]{Threads}

...

\clearpage

{\color{insertcolor}

%******************************************************************
\rSec1[thread.stop_token]{Stop Tokens}

\pnum
\ref{thread.stop_token} describes components that can be used to
asynchonously request an end ("stop") of a running execution.
The stop can only be requested exactly once
by one of multiple \tcode{stop_source}s to one or multiple \tcode{stop_token}s.
Callbacks can be registered as \tcode{stop_callback}s to be called when the stop is requested.

For this, classes \tcode{stop_source}, \tcode{stop_token} and \tcode{stop_callback} implement
semantics of shared ownership of an associated atomic stop state (an atomic token to signal a stop request).
The last remaining owner of the stop state automatically 
releases the resources associated with the stop state.

\pnum
Calls to \tcode{request_stop()}, \tcode{stop_requested()},
\tcode{stoppable()}, and \tcode{callbacks_ignored()}
are atomic operations (6.8.2.1p3 \ref{intro.races})
on the stop state contained in the stop state object.
Hence concurrent calls to these functions do not introduce data races. 
A call to \tcode{request_stop()} synchronizes with any call to \tcode{request_stop()} and
\tcode{stop_requested()} that observes the stop.
%(and hence returns \tcode{true} or throws).

%**************************
\rSec2[thread.stop_token.syn]{Header \tcode{<stop_token>} synopsis}
\indexhdr{stop_token}%

\begin{codeblock}
namespace std {
  // \ref{stop_token} class \tcode{stop_token}
  template <typename Callback> class stop_callback;
  class stop_source;
  class stop_token;
}
\end{codeblock}


%**************************
% stop_callback
%**************************
\indexlibrary{\idxcode{stop_callback}}%
\rSec2[stop_callback]{Class \tcode{stop_callback}}

\pnum
\indexlibrary{\idxcode{stop_callback}}%

\begin{codeblock}
namespace std {
  template <Invocable Callback>
    requires MoveConstructible<Callback>
  class stop_callback {
  public:
    // \ref{stop_callback.constr} create, destroy:
    explicit stop_callback(const stop_token& st, Callback&& cb)
        noexcept(std::is_nothrow_move_constructible_v<Callback>);
    explicit stop_callback(stop_token&& st, Callback&& cb)
        noexcept(std::is_nothrow_move_constructible_v<Callback>);
    ~stop_callback();

    stop_callback(const stop_callback&) = delete;
    stop_callback(stop_callback&&) = delete;
    stop_callback& operator=(const stop_callback&) = delete;
    stop_callback& operator=(stop_callback&&) = delete;

  private:
    // \expos
    Callback callback; 
  };

  template <typename Callback>
  stop_callback(const stop_token&, Callback&&) -> stop_callback<Callback>;

  template <typename Callback>
  stop_callback(stop_token&&, Callback&&) -> stop_callback<Callback>;
}
\end{codeblock}

%*****
\rSec3[stop_callback.constr]{\tcode{stop_callback} constructors and destructor}

\indexlibrary{\idxcode{stop_callback}!constructor}%
\begin{itemdecl}
explicit stop_callback(const stop_token& st, Callback&& cb)
  noexcept(std::is_nothrow_move_constructible_v<Callback>);
explicit stop_callback(stop_token&& st, Callback&& cb)
  noexcept(std::is_nothrow_move_constructible_v<Callback>);
\end{itemdecl}
\begin{itemdescr}
  \pnum\effects Initialises \tcode{callback} with \tcode{static_cast<Callback\&\&>(cb)}.
                If \tcode{it.stop_requested()} is true then immediately invokes \tcode{static_cast<Callback\&\&>(callback)}
                with zero arguments on the current thread before the constructor returns.
                Otherwise, the callback is registered with the shared stop state of \tcode{it}
                such that \tcode{static_cast<Callback\&\&>(callback)} is invoked by first call to \tcode{isrc.request_stop()}
                on an \tcode{stop_source} instance \tcode{isrc} that references the same shared stop
                state as \tcode{it}.
                If invoking the callback throws an unhandled exception then \tcode{std::terminate()} is called.
\end{itemdescr}

\indexlibrary{\idxcode{stop_callback}!destructor}%
\begin{itemdecl}
~stop_callback();
\end{itemdecl}
\begin{itemdescr}
  \pnum\effects Deregisters the callback from the associated stop state.
                If this callback is concurrently executing on another thread then the destructor shall
                block until the callback returns before calling \tcode{callback}'s destructor.
                The destructor shall not block waiting for the execution of another callback registered
                with the same shared stop state to finish.
                A subsequent call to \tcode{isrc.request_stop()} on an \tcode{stop_source}, \tcode{isrc}, with the same
                associated stop state shall not invoke the callback once the destructor has returned.
\end{itemdescr}

%**************************
% stop_source
%**************************
\indexlibrary{\idxcode{stop_source}}%
\rSec2[stop_source]{Class \tcode{stop_source}}

\pnum
\indexlibrary{\idxcode{stop_token}}%
The class \tcode{stop_source} implements semantics of signaling stop requests
to \tcode{stop_token}s (\ref{stop_token}) sharing the same associated stop state.
All \tcode{stop_source}s sharing the same stop state can request a stop.
An stop can only be requested once. Subsequent attempts to request a stop are no-ops.

\begin{codeblock}
namespace std {
  class stop_source {
  public:
    // \ref{stop_source.constr} create, copy, destroy:
    stop_source();
    explicit stop_source(nullptr_t) noexcept;

    stop_source(const stop_source&) noexcept;
    stop_source(stop_source&&) noexcept;
    stop_source& operator=(const stop_source&) noexcept;
    stop_source& operator=(stop_source&&) noexcept;
    ~stop_source();
    void swap(stop_source&) noexcept;

    // \ref{stop_source.mem} stop handling:
    [[nodiscard]] stop_token get_token() const noexcept;
    [[nodiscard]] bool stoppable() const noexcept;
    [[nodiscard]] bool stop_requested() const noexcept;
    [[nodiscard]] bool request_stop() const noexcept;

    friend bool operator== (const stop_source& lhs, const stop_source& rhs) noexcept;
    friend bool operator!= (const stop_source& lhs, const stop_source& rhs) noexcept;
  };
}
\end{codeblock}

%***** constructors
\rSec3[stop_source.constr]{\tcode{stop_source} constructors}

\indexlibrary{\idxcode{stop_source}!constructor}%
\begin{itemdecl}
stop_source();
\end{itemdecl}
\begin{itemdescr}
  \pnum\effects Constructs a new \tcode{stop_source} object that can be used to request stops.
  
  \pnum\postconditions \tcode{stoppable() == true} and \tcode{stop_requested() == false}.

  \pnum\throws \tcode{bad_alloc} If memory could not be allocated for the shared stop state.
\end{itemdescr}

\indexlibrary{\idxcode{stop_source}!constructor}%
\begin{itemdecl}
explicit stop_source(nullptr_t) noexcept;
\end{itemdecl}
\begin{itemdescr}
  \pnum\effects Constructs a new \tcode{stop_source} object that can't be used to request stops.
                \begin{note} Therefore, no resources have to be associated for the state.  \end{note}

  \pnum\postconditions \tcode{stoppable() == false}.
\end{itemdescr}

%*** special members:

\indexlibrary{\idxcode{stop_source}!constructor}%
\begin{itemdecl}
stop_source(const stop_source& rhs) noexcept;
\end{itemdecl}
\begin{itemdescr}
  \pnum\effects If \tcode{rhs.stoppable() == true},
                constructs an \tcode{stop_source}
                that shares the ownership of the stop state with \tcode{rhs}.

  \pnum\postconditions \tcode{stoppable() == rhs.stoppable()}
                and \tcode{stop_requested() == rhs.stop_requested()}
                and \tcode{*this == rhs}.
\end{itemdescr}

\indexlibrary{\idxcode{stop_source}!constructor}%
\begin{itemdecl}
stop_source(stop_source&& rhs) noexcept;
\end{itemdecl}
\begin{itemdescr}
  \pnum\effects Move constructs an object of type \tcode{stop_source} from \tcode{rhs}.

  \pnum\postconditions \tcode{*this} shall contain the old value of \tcode{rhs} and
                        \tcode{rhs.stoppable() == false}.
\end{itemdescr}

%*****
\rSec3[stop_source.destr]{\tcode{stop_source} destructor}

\indexlibrary{\idxcode{stop_source}!destructor}%
\begin{itemdecl}
~stop_source();
\end{itemdecl}

\begin{itemdescr}
 \pnum\effects If \tcode{stoppable()} and \tcode{*this} is the last owner of the stop state,
                releases the resources associated with the stop state.
\end{itemdescr}

%*****
\rSec3[stop_source.assign]{\tcode{stop_source} assignment}

\indexlibrarymember{operator=}{stop_source}%
\begin{itemdecl}
stop_source& operator=(const stop_source& rhs) noexcept;
\end{itemdecl}
\begin{itemdescr}
  \pnum\effects Equivalent to: \tcode{stop_source(rhs).swap(*this);}

  \pnum\returns \tcode{*this}.
\end{itemdescr}

\indexlibrarymember{operator=}{stop_source}%
\begin{itemdecl}
stop_source& operator=(stop_source&& rhs) noexcept;
\end{itemdecl}
\begin{itemdescr}
  \pnum\effects Equivalent to: \tcode{stop_source(std::move(rhs)).swap(*this);}

  \pnum\returns \tcode{*this}.
\end{itemdescr}

%*****
\rSec3[stop_source.swap]{\tcode{stop_source} swap}

\indexlibrarymember{swap}{stop_source}%
\begin{itemdecl}
void swap(stop_source& rhs) noexcept;
\end{itemdecl}

\begin{itemdescr}
 \pnum \effects Swaps the state of \tcode{*this} and \tcode{rhs}.
\end{itemdescr}


%***** get_token()
\rSec3[stop_source.mem]{\tcode{stop_source} members}

\indexlibrarymember{get_token}{stop_source}%
\begin{itemdecl}
[[nodiscard]] stop_token get_token() const noexcept;
\end{itemdecl}
\begin{itemdescr}
  \pnum\effects If \tcode{!stoppable()}, constructs an \tcode{stop_token} object
                that does not share a stop state.
                Otherwise, constructs an \tcode{stop_token} object \tcode{it}
                that shares the ownership of the stop state with \tcode{*this}.

  \pnum\postconditions \tcode{stoppable() != it.callbacks_ignored()}
                and \tcode{stop_requested() == it.stop_requested()}.
\end{itemdescr}


\indexlibrarymember{stoppable}{stop_source}%
\begin{itemdecl}
[[nodiscard]] bool stoppable() const noexcept;
\end{itemdecl}
\begin{itemdescr}
  \pnum\returns \tcode{true} if the stop source can be used to request stops.
                \begin{note} Returns \tcode{false} if the object was created with the \tcode{nullptr}
                             or the values were moved away.
                             \end{note}
\end{itemdescr}

\indexlibrarymember{stop_requested}{stop_source}%
\begin{itemdecl}
[[nodiscard]] bool stop_requested() const noexcept;
\end{itemdecl}
\begin{itemdescr}
  \pnum\returns \tcode{true} if \tcode{stoppable()} 
                and \tcode{request_stop()} was called by one of the owners.
\end{itemdescr}

\indexlibrarymember{request_stop}{stop_source}%
\begin{itemdecl}
[[nodiscard]] bool request_stop() const noexcept;
\end{itemdecl}
\begin{itemdescr}
  %\pnum\requires \tcode{stoppable() == true}
  %Hans:
  %Did we discuss whether interrupt() on an invalid (e.g. default constructed) stop/interrupt token should be a no-op,
  %rather than undefined? I would expect that to be convenient if we start passing stop_tokens around explicitly.
  %Presumably a lot of short-lived tasks don't care about stopping/interruption, but might be used in a context in which an
  %stop/interrupt token is expected. With the change, you could just pass a default-constructed one.

  \pnum\effects If \tcode{!stoppable()} or \tcode{stop_requested()} the call has no effect. 
                Otherwise, requests a stop so that \tcode{stop_requested() == true}
                and all registered callbacks are synchronously called.
                \begin{note} Requesting a stop includes notifying all condition variables
                             of type \tcode{condition_variable_any}
                             temporarily registered during an interruptable wait (\ref{thread.condition.stop_source})
                             \end{note}

  \pnum\postconditions \tcode{!stoppable() || stop_requested()}

  \pnum\returns The value of \tcode{stop_requested()} prior to the call.
\end{itemdescr}


%*****
\rSec3[stop_source.cmp]{\tcode{stop_source} comparisons}

\indexlibrarymember{operator==}{stop_source}%
\begin{itemdecl}
bool operator== (const stop_source& lhs, const stop_source& rhs) noexcept;
\end{itemdecl}
\begin{itemdescr}
  \pnum\returns \tcode{!lhs.stoppable() \&\& !rhs.stoppable()} or
                whether \tcode{lhs} and \tcode{rhs} refer to the
                same stop state
                (copied or moved from the same initial \tcode{stop_source} object).
\end{itemdescr}

\indexlibrarymember{operator!=}{stop_source}%
\begin{itemdecl}
bool operator!= (const stop_source& lhs, const stop_source& rhs) noexcept;
\end{itemdecl}
\begin{itemdescr}
  \pnum\returns \tcode{!(lhs==rhs)}.
\end{itemdescr}


%**************************
% stop_token
%**************************
\indexlibrary{\idxcode{stop_token}}%
\rSec2[stop_token]{Class \tcode{stop_token}}

\pnum
\indexlibrary{\idxcode{stop_token}}%
The class \tcode{stop_token} provides an interface for responding to stops requested from
the \tcode{stop_source} object they were created from.
All tokens can check whether an stop was requested.
When an stop is requested, which is possible only once,
any registered \tcode{stop_callback} (\ref{stop_callback}) is called.
Registering a callback after an stop was already requested calls the callback immediately.

\begin{codeblock}
namespace std {
  class stop_token {
  public:
    // \ref{stop_token.constr} create, copy, destroy:
    explicit stop_token() noexcept;

    stop_token(const stop_token&) noexcept;
    stop_token(stop_token&&) noexcept;
    stop_token& operator=(const stop_token&) noexcept;
    stop_token& operator=(stop_token&&) noexcept;
    ~stop_token();
    void swap(stop_token&) noexcept;

    // \ref{stop_token.mem} stop handling:
    [[nodiscard]] bool stop_requested() const noexcept;
    [[nodiscard]] bool callbacks_ignored() const noexcept;

    friend bool operator== (const stop_token& lhs, const stop_token& rhs) noexcept;
    friend bool operator!= (const stop_token& lhs, const stop_token& rhs) noexcept;
  };
}
\end{codeblock}


%*****
\rSec3[stop_token.constr]{\tcode{stop_token} constructors}

\indexlibrary{\idxcode{stop_token}!constructor}%
\begin{itemdecl}
stop_token() noexcept;
\end{itemdecl}
\begin{itemdescr}
  \pnum\effects Constructs a new \tcode{stop_token} object that can't be used to request stops.
                \begin{note} Therefore, no resources have to be associated for the state.  \end{note}

  \pnum\postconditions \tcode{callbacks_ignored() == true} and 
                       \tcode{stop_requested() == false}.
\end{itemdescr}

%***** special members:

\indexlibrary{\idxcode{stop_token}!constructor}%
\begin{itemdecl}
stop_token(const stop_token& rhs) noexcept;
\end{itemdecl}
\begin{itemdescr}
  \pnum\effects If \tcode{rhs.callbacks_ignored() == true}, constructs an \tcode{stop_token} object
                that can't be used to request stops.
                Otherwise, constructs an \tcode{stop_token}
                that shares the ownership of the stop state with \tcode{rhs}.

  \pnum\postconditions \tcode{callbacks_ignored() == rhs.callbacks_ignored()}
                and \tcode{stop_requested() == rhs.stop_requested()}
                and \tcode{*this == rhs}.
\end{itemdescr}

\indexlibrary{\idxcode{stop_token}!constructor}%
\begin{itemdecl}
stop_token(stop_token&& rhs) noexcept;
\end{itemdecl}
\begin{itemdescr}
  \pnum\effects Move constructs an object of type \tcode{stop_token} from \tcode{rhs}.

  \pnum\postconditions \tcode{*this} shall contain the old value of \tcode{rhs} and
                        \tcode{rhs.callbacks_ignored() == true}.
\end{itemdescr}

%*****
\rSec3[stop_token.destr]{\tcode{stop_token} destructor}

\indexlibrary{\idxcode{stop_token}!destructor}%
\begin{itemdecl}
~stop_token();
\end{itemdecl}

\begin{itemdescr}
 \pnum\effects If \tcode{*this} is the last owner of the stop state,
                releases the resources associated with the stop state.
\end{itemdescr}

%*****
\rSec3[stop_token.assign]{\tcode{stop_token} assignment}

\indexlibrarymember{operator=}{stop_token}%
\begin{itemdecl}
stop_token& operator=(const stop_token& rhs) noexcept;
\end{itemdecl}
\begin{itemdescr}
  \pnum\effects Equivalent to: \tcode{stop_token(rhs).swap(*this);}

  \pnum\returns \tcode{*this}.
\end{itemdescr}

\indexlibrarymember{operator=}{stop_token}%
\begin{itemdecl}
stop_token& operator=(stop_token&& rhs) noexcept;
\end{itemdecl}
\begin{itemdescr}
  \pnum\effects Equivalent to: \tcode{stop_token(std::move(rhs)).swap(*this);}

  \pnum\returns \tcode{*this}.
\end{itemdescr}

%*****
\rSec3[stop_token.swap]{\tcode{stop_token} swap}

\indexlibrarymember{swap}{stop_token}%
\begin{itemdecl}
void swap(stop_token& rhs) noexcept;
\end{itemdecl}

\begin{itemdescr}
\pnum
\effects Swaps the state of \tcode{*this} and \tcode{rhs}.
\end{itemdescr}


%*****
\rSec3[stop_token.mem]{\tcode{stop_token} members}

\indexlibrarymember{stop_requested}{stop_token}%
\begin{itemdecl}
[[nodiscard]] bool stop_requested() const noexcept;
\end{itemdecl}
\begin{itemdescr}
  \pnum\returns \tcode{true} if \tcode{callbacks_ignored() == false()} 
                and \tcode{request_stop()} was called by one of the owners,
                otherwise \tcode{false}.
  \pnum\sync If \tcode{true} is returned then synchronizes with the
             first call to \tcode{request_stop()} by one of the owners.
\end{itemdescr}

\indexlibrarymember{callbacks_ignored}{stop_token}%
\begin{itemdecl}
[[nodiscard]] bool callbacks_ignored() const noexcept;
\end{itemdecl}
\begin{itemdescr}
  \pnum\returns \tcode{true} if the stop token will never call any (newly) registered callbacks.
                \begin{note}To return \tcode{false} (i.e., to be able to call (new) callbacks), 
                  a \tcode{stop_token} must share a stop state, for which either a stop already was requested
                  or still a \tcode{stop_source} exists that can potentially
                  be used to call \tcode{request_stop()}.
                  Thus, if it returns \tcode{true}, it makes no sense to register a callback.
                \end{note}
\end{itemdescr}


%*****
\rSec3[stop_token.cmp]{\tcode{stop_token} comparisons}

\indexlibrarymember{operator==}{stop_token}%
\begin{itemdecl}
bool operator== (const stop_token& lhs, const stop_token& rhs) noexcept;
\end{itemdecl}
\begin{itemdescr}
  \pnum\returns \tcode{lhs.callbacks_ignored() \&\& rhs.callbacks_ignored()} or
                whether \tcode{lhs} and \tcode{rhs} refer to the
                same stop state
                (copied or moved from the same initial \tcode{stop_source} object).
\end{itemdescr}

\indexlibrarymember{operator!=}{stop_token}%
\begin{itemdecl}
bool operator!= (const stop_token& lhs, const stop_token& rhs) noexcept;
\end{itemdecl}
\begin{itemdescr}
  \pnum\returns \tcode{!(lhs==rhs)}.
\end{itemdescr}


\clearpage

%******************************************************************
\rSec1[thread.jthreads]{Joining Threads}


\pnum
\ref{thread.jthreads} describes components that can be used to create and manage threads
with the ability to request stops to cooperatively cancel the running thread.

%**************************
\rSec2[thread.jthread.syn]{Header \tcode{<jthread>} synopsis}
\indexhdr{jthread}%

\begin{codeblock}
#include <stop_token>

namespace std {
  // \ref{thread.jthread.class} class \tcode{jthread}
  class jthread;

  void swap(jthread& x, jthread& y) noexcept;
}
\end{codeblock}


%**************************
\rSec2[thread.jthread.class]{Class \tcode{jthread}}

\pnum
The class \tcode{jthread} provides a mechanism
to create a new thread of execution.
The functionality is the same as for class \tcode{thread} (\ref{thread.thread.class})
with the additional ability to request a stop and to
automatically \tcode{join()} the started thread.

{\color{blue}
[{\itshape{}Editorial note:} {\color{diffcolor}This color signals differences to class \tcode{std::thread}.} ]
}

\indexlibrary{\idxcode{jthread}}%
\begin{codeblock}
namespace std {
  class jthread {
  public:
    // types
    using id = thread::id;
    using native_handle_type = thread::native_handle_type;

    // construct/copy/destroy
    jthread() noexcept;
    template<class F, class... Args> explicit jthread(F&& f, Args&&... args);
    ~jthread();
    jthread(const jthread&) = delete;
    jthread(jthread&&) noexcept;
    jthread& operator=(const jthread&) = delete;
    jthread& operator=(jthread&&) noexcept;

    // members
    void swap(jthread&) noexcept;
    bool joinable() const noexcept;
    void join();
    void detach();
    [[nodiscard]] id get_id() const noexcept;
    [[nodiscard]] native_handle_type native_handle();     // see~\ref{thread.req.native}
\end{codeblock}
{\color{diffcolor}
\begin{codeblock}
    // stop token handling
    [[nodiscard]] stop_token get_stop_source() const noexcept;
    [[nodiscard]] bool request_stop() noexcept;
\end{codeblock}
}%\color{diffcolor}
\begin{codeblock}
    // static members
    [[nodiscard]] static unsigned int hardware_concurrency() noexcept;
  
\end{codeblock}
{\color{diffcolor}
\begin{codeblock}
  private:
    stop_token isource;                 // \expos
  };
\end{codeblock}
}%\color{diffcolor}
\begin{codeblock}
}
\end{codeblock}


\rSec3[thread.jthread.constr]{\tcode{jthread} constructors}

\indexlibrary{\idxcode{jthread}!constructor}%
\begin{itemdecl}
jthread() noexcept;
\end{itemdecl}
\begin{itemdescr}
  \pnum\effects Constructs a \tcode{jthread} object that does not represent a thread of execution.

  \pnum\postconditions \tcode{get_id() == id()}
        {\color{diffcolor} and \tcode{isource.stoppable() == false}}.
\end{itemdescr}

\indexlibrary{\idxcode{jthread}!constructor}%
\begin{itemdecl}
template<class F, class... Args> explicit jthread(F&& f, Args&&... args);
\end{itemdecl}
\begin{itemdescr}
  \pnum
\requires\ \tcode{F} and each $\tcode{T}_i$ in \tcode{Args} shall satisfy the
\oldconcept{MoveConstructible} requirements.
{\color{diffcolor}
        \tcode{%
                \placeholdernc{INVOKE}(\brk{}%
                \placeholdernc{DECAY_COPY}(\brk{}%
                std::forward<F>(f)),
                isource,
                \placeholdernc{DECAY_COPY}(\brk{}%
                std::forward<Args>(\brk{}args))...)}
        or
}
        \tcode{%
                \placeholdernc{INVOKE}(\brk{}%
                \placeholdernc{DECAY_COPY}(\brk{}%
                std::forward<F>(f)),
                \placeholdernc{DECAY_COPY}(\brk{}%
                std::forward<Args>(\brk{}args))...)}
\iref{func.require} shall be a valid expression.

\pnum\remarks
This constructor shall not participate in overload resolution if \tcode{remove_cvref_t<F>}
is the same type as \tcode{std::jthread}.

\pnum\effects
{\color{diffcolor} Initializes \tcode{isource} and
}
constructs an object of type \tcode{jthread}.
The new thread of execution executes
{\color{diffcolor}
        \tcode{%
                \placeholdernc{INVOKE}(\brk{}%
                \placeholdernc{DECAY_COPY}(\brk{}%
                std::forward<F>(f)),
                isource,%
                \placeholdernc{DECAY_COPY}(\brk{}%
                std::forward<Args>(\brk{}args))...)}
if that expression is well-formed,
otherwise
}
        \tcode{%
                \placeholdernc{INVOKE}(\brk{}%
                \placeholdernc{DECAY_COPY}(\brk{}%
                std::forward<F>(f)),
                \placeholdernc{DECAY_COPY}(\brk{}%
                std::forward<Args>(\brk{}args))...)}
with the calls to
\tcode{\placeholder{DECAY_COPY}} being evaluated in the constructing thread.
Any return value from this invocation is ignored.
\begin{note} This implies that any exceptions not thrown from the invocation of the copy
of \tcode{f} will be thrown in the constructing thread, not the new thread. \end{note}
If the invocation
with \tcode{\placeholdernc{INVOKE}()}
%of
%\tcode{%
%\placeholdernc{INVOKE}(\brk{}%
%\placeholdernc{DECAY_COPY}(\brk{}%
%std::forward<F>(f)),
%\placeholdernc{DECAY_COPY}(\brk{}%
%std::forward<Args>(args))...)}
terminates with an uncaught exception, \tcode{terminate()} shall be called.

\pnum\sync The completion of the invocation of the constructor
synchronizes with the beginning of the invocation of the copy of \tcode{f}.

\pnum\postconditions \tcode{get_id() != id()}.
{\color{diffcolor}
                     \tcode{isource.stoppable() == true}.
}
                     \tcode{*this} represents the newly started thread.
{\color{diffcolor}
\begin{note} Note that the calling thread can request a stop only once,
                because it can't replace this stop token.  \end{note}
}%\color{diffcolor}

\pnum\throws \tcode{system_error} if unable to start the new thread.

\pnum\errors
\begin{itemize}
\item \tcode{resource_unavailable_try_again} --- the system lacked the necessary
resources to create another thread, or the system-imposed limit on the number of
threads in a process would be exceeded.
\end{itemize}
\end{itemdescr}

\indexlibrary{\idxcode{jthread}!constructor}%
\begin{itemdecl}
jthread(jthread&& x) noexcept;
\end{itemdecl}

\begin{itemdescr}
\pnum
\effects Constructs an object of type \tcode{jthread} from \tcode{x}, and sets
\tcode{x} to a default constructed state.

\pnum
\postconditions \tcode{x.get_id() == id()} and \tcode{get_id()} returns the
value of \tcode{x.get_id()} prior to the start of construction.
{\color{diffcolor}
\tcode{isource} yields the value of \tcode{x.isource} prior to the start of construction
and \tcode{x.isource.stoppable() == false}.
}%\color{diffcolor}

\end{itemdescr}

\rSec3[thread.jthread.destr]{\tcode{jthread} destructor}

\indexlibrary{\idxcode{jthread}!destructor}%
\begin{itemdecl}
~jthread();
\end{itemdecl}

{\color{diffcolor}
\begin{itemdescr}
\pnum
If \tcode{joinable()}, calls \tcode{request_stop()} and \tcode{join()}.
Otherwise, has no effects.
\begin{note} Operations on \tcode{*this} are not synchronized. \end{note}
\end{itemdescr}
}%\color{diffcolor}

\rSec3[thread.jthread.assign]{\tcode{jthread} assignment}

\indexlibrarymember{operator=}{jthread}%
\begin{itemdecl}
jthread& operator=(jthread&& x) noexcept;
\end{itemdecl}

\begin{itemdescr}
\pnum
\effects If \tcode{joinable()}, calls \tcode{request_stop()} and \tcode{join()}.
Assigns the
state of \tcode{x} to \tcode{*this} and sets \tcode{x} to a default constructed state.

{\color{diffcolor}
\pnum
\postconditions \tcode{x.get_id() == id()} and \tcode{get_id()} returns the value of
\tcode{x.get_id()} prior to the assignment.
\tcode{isource} yields the value of \tcode{x.isource} prior to the assignment
and \tcode{x.isource.stoppable() == false}.
}%\color{diffcolor}

\pnum
\returns \tcode{*this}.
\end{itemdescr}


\rSec3[thread.jthread.stop]{\tcode{jthread} stop members}

{\color{diffcolor}
\indexlibrarymember{get_stop_source}{jthread}%
\begin{itemdecl}
[[nodiscard]] stop_token get_stop_source() const noexcept
\end{itemdecl}
\begin{itemdescr}
  \pnum\effects Equivalent to: \tcode{return isource;}
\end{itemdescr}

\indexlibrarymember{request_stop}{jthread}%
\begin{itemdecl}
[[nodiscard]] bool request_stop() noexcept;
\end{itemdecl}
\begin{itemdescr}
  \pnum\effects Equivalent to: \tcode{return isource.request_stop();}
\end{itemdescr}
}%\color{diffcolor}

%\rSec3[thread.jthread.static]{\tcode{jthread} static members}
%
%\indexlibrarymember{hardware_concurrency}{thread}%
%\begin{itemdecl}
%unsigned hardware_concurrency() noexcept;
%\end{itemdecl}
%\begin{itemdescr}
%  \pnum\effects Equivalent to: \tcode{return thread::hardware_concurrency()}.
%\end{itemdescr}
%
%\rSec3[thread.jthread.algorithm]{\tcode{jthread} specialized algorithms}
%
%\indexlibrarymember{swap}{jthread}%
%\begin{itemdecl}
%void swap(jthread& x, jthread& y) noexcept;
%\end{itemdecl}
%\begin{itemdescr}
%  \pnum\effects As if by \tcode{x.swap(y)}.
%\end{itemdescr}

}%\color{insertcolor}


\clearpage

%******************************************************************
\rSec1[thread.mutex]{Mutual exclusion}

...

%******************************************************************
\rSec1[thread.condition]{Condition variables}

...

\rSec2[condition_variable.syn]{Header \tcode{<condition_variable>} synopsis}

...

\rSec2[thread.condition.nonmember]{Non-member functions}

...

\rSec2[thread.condition.condvar]{Class \tcode{condition_variable}}

...

\rSec2[thread.condition.condvarany]{Class \tcode{condition_variable_any}}

...

\indexlibrary{\idxcode{condition_variable_any}}%
\begin{codeblock}
namespace std {
  class condition_variable_any {
  public:
    condition_variable_any();
    ~condition_variable_any();

    condition_variable_any(const condition_variable_any&) = delete;
    condition_variable_any& operator=(const condition_variable_any&) = delete;

    void notify_one() noexcept;
    void notify_all() noexcept;

\end{codeblock}
{\color{insertcolor}
\begin{codeblock}
    // \ref{thread.condvarany.wait} noninterruptable waits:
\end{codeblock}
}%insertcolor
\begin{codeblock}
    template<class Lock>
      void wait(Lock& lock);
    template<class Lock, class Predicate>
      void wait(Lock& lock, Predicate pred);

    template<class Lock, class Clock, class Duration>
      cv_status wait_until(Lock& lock, const chrono::time_point<Clock, Duration>& abs_time);
    template<class Lock, class Clock, class Duration, class Predicate>
      bool wait_until(Lock& lock, const chrono::time_point<Clock, Duration>& abs_time,
                      Predicate pred);
    template<class Lock, class Rep, class Period>
      cv_status wait_for(Lock& lock, const chrono::duration<Rep, Period>& rel_time);
    template<class Lock, class Rep, class Period, class Predicate>
      bool wait_for(Lock& lock, const chrono::duration<Rep, Period>& rel_time, Predicate pred);

\end{codeblock}
{\color{insertcolor}
\begin{codeblock}
    // \ref{thread.condvarany.interruptwait} \tcode{stop_token} waits:
    template <class Lock, class Predicate>
      bool wait_until(Lock& lock,
                      Predicate pred,
                      stop_token stoken);
    template <class Lock, class Clock, class Duration, class Predicate>
      bool wait_until(Lock& lock,
                      const chrono::time_point<Clock, Duration>& abs_time
                      Predicate pred,
                      stop_token stoken);
    template <class Lock, class Rep, class Period, class Predicate>
      bool wait_for(Lock& lock,
                    const chrono::duration<Rep, Period>& rel_time,
                    Predicate pred,
                    stop_token stoken);
\end{codeblock}
}
\begin{codeblock}
  };
}
\end{codeblock}


\indexlibrary{\idxcode{condition_variable_any}!constructor}%
\begin{itemdecl}
condition_variable_any();
\end{itemdecl}

\begin{itemdescr}
\pnum
\effects Constructs an object of type \tcode{condition_variable_any}.

\pnum
\throws \tcode{bad_alloc} or \tcode{system_error} when an exception is
required\iref{thread.req.exception}.

\pnum
\errors
\begin{itemize}
\item \tcode{resource_unavailable_try_again} --- if some non-memory resource
limitation prevents initialization.

\item \tcode{operation_not_permitted} --- if the thread does not have the
privilege to perform the operation.
\end{itemize}
\end{itemdescr}

\indexlibrary{\idxcode{condition_variable_any}!destructor}%
\begin{itemdecl}
~condition_variable_any();
\end{itemdecl}

\begin{itemdescr}
\pnum
\requires There shall be no thread blocked on \tcode{*this}. \begin{note} That is, all
threads shall have been notified; they may subsequently block on the lock specified in the
wait.
This relaxes the usual rules, which would have required all wait calls to happen before
destruction. Only the notification to unblock the wait needs to happen before destruction.
The user should take care to ensure that no threads wait on \tcode{*this} once the destructor has
been started, especially when the waiting threads are calling the wait functions in a loop or
using the overloads of \tcode{wait}, \tcode{wait_for}, or \tcode{wait_until} that take a predicate.
\end{note}

\pnum\effects Destroys the object.
\end{itemdescr}

\indexlibrarymember{notify_one}{condition_variable_any}%
\begin{itemdecl}
void notify_one() noexcept;
\end{itemdecl}

\begin{itemdescr}
\pnum\effects If any threads are blocked waiting for \tcode{*this}, unblocks one of those threads.
\end{itemdescr}

\indexlibrarymember{notify_all}{condition_variable_any}%
\begin{itemdecl}
void notify_all() noexcept;
\end{itemdecl}

\begin{itemdescr}
\pnum\effects Unblocks all threads that are blocked waiting for \tcode{*this}.
\end{itemdescr}


%**************************************
%*** wait functions:

{\color{insertcolor}
\rSec3[thread.condvarany.wait]{Noninterruptable waits}
}

% wait(Lock)
\indexlibrarymember{wait}{condition_variable_any}%
\begin{itemdecl}
template<class Lock>
  void wait(Lock& lock);
\end{itemdecl}

\begin{itemdescr}
\pnum
\effects
\begin{itemize}
\item Atomically calls \tcode{lock.unlock()} and blocks on \tcode{*this}.
\item When unblocked, calls \tcode{lock.lock()} (possibly blocking on the lock) and returns.
\item The function will unblock when requested by a call to \tcode{notify_one()},
a call to \tcode{notify_all()}, or spuriously.
\end{itemize}

\pnum
\remarks
If the function fails to meet the postcondition, \tcode{terminate()}
shall be called\iref{except.terminate}.
\begin{note} This can happen if the re-locking of the mutex throws an exception. \end{note}

\pnum\postconditions \tcode{lock} is locked by the calling thread.

\pnum\throws Nothing.

\end{itemdescr}


% wait(Lock, Predicate)
\indexlibrarymember{wait}{condition_variable_any}%
\begin{itemdecl}
template<class Lock, class Predicate>
  void wait(Lock& lock, Predicate pred);
\end{itemdecl}

\begin{itemdescr}
\pnum
\effects Equivalent to:
\begin{codeblock}
while (!pred())
  wait(lock);
\end{codeblock}
\end{itemdescr}


% wait_until(Lock, Timepoint)
\indexlibrarymember{wait_until}{condition_variable_any}%
\begin{itemdecl}
template<class Lock, class Clock, class Duration>
  cv_status wait_until(Lock& lock, const chrono::time_point<Clock, Duration>& abs_time);
\end{itemdecl}

\begin{itemdescr}
\pnum\effects

\begin{itemize}
\item
Atomically calls \tcode{lock.unlock()} and blocks on \tcode{*this}.

\item
When unblocked, calls \tcode{lock.lock()} (possibly blocking on the lock) and returns.

\item
The function will unblock when requested by a call to \tcode{notify_one()}, a call to \tcode{notify_all()},
expiration of the absolute timeout\iref{thread.req.timing} specified by \tcode{abs_time},
or spuriously.

\item
If the function exits via an exception, \tcode{lock.lock()} shall be called prior to exiting the function.
\end{itemize}

\pnum
\remarks
If the function fails to meet the postcondition, \tcode{terminate()}
shall be called\iref{except.terminate}.
\begin{note} This can happen if the re-locking of the mutex throws an exception. \end{note}

\pnum
\postconditions \tcode{lock} is locked by the calling thread.

\pnum
\returns \tcode{cv_status::timeout} if
the absolute timeout\iref{thread.req.timing} specified by \tcode{abs_time} expired,
otherwise \tcode{cv_status::no_timeout}.

\pnum
\throws Timeout-related
exceptions\iref{thread.req.timing}.

\end{itemdescr}

\indexlibrarymember{wait_for}{condition_variable_any}%
\begin{itemdecl}
template<class Lock, class Rep, class Period>
  cv_status wait_for(Lock& lock, const chrono::duration<Rep, Period>& rel_time);
\end{itemdecl}

\begin{itemdescr}
\pnum
\effects Equivalent to:
\begin{codeblock}
return wait_until(lock, chrono::steady_clock::now() + rel_time);
\end{codeblock}

\pnum
\returns \tcode{cv_status::timeout} if
the relative timeout\iref{thread.req.timing} specified by \tcode{rel_time} expired,
otherwise \tcode{cv_status::no_timeout}.

\pnum
\remarks
If the function fails to meet the postcondition, \tcode{terminate()}
shall be called\iref{except.terminate}.
\begin{note} This can happen if the re-locking of the mutex throws an exception. \end{note}

\pnum
\postconditions \tcode{lock} is locked by the calling thread.

\pnum
\throws Timeout-related
exceptions\iref{thread.req.timing}.

\end{itemdescr}

\indexlibrarymember{wait_until}{condition_variable_any}%
\begin{itemdecl}
template<class Lock, class Clock, class Duration, class Predicate>
  bool wait_until(Lock& lock, const chrono::time_point<Clock, Duration>& abs_time, Predicate pred);
\end{itemdecl}

\begin{itemdescr}
\pnum
\effects Equivalent to:
\begin{codeblock}
while (!pred())
  if (wait_until(lock, abs_time) == cv_status::timeout)
    return pred();
return true;
\end{codeblock}

\pnum
\begin{note} There is no blocking if \tcode{pred()} is initially \tcode{true}, or
if the timeout has already expired. \end{note}

\pnum
\begin{note} The returned value indicates whether the predicate evaluates to \tcode{true}
regardless of whether the timeout was triggered. \end{note}
\end{itemdescr}

\indexlibrarymember{wait_for}{condition_variable_any}%
\begin{itemdecl}
template<class Lock, class Rep, class Period, class Predicate>
  bool wait_for(Lock& lock, const chrono::duration<Rep, Period>& rel_time, Predicate pred);
\end{itemdecl}

\begin{itemdescr}
\pnum
\effects Equivalent to:
\begin{codeblock}
return wait_until(lock, chrono::steady_clock::now() + rel_time, std::move(pred));
\end{codeblock}
\end{itemdescr}



\clearpage

%**********************************************************
%\begin{codeblock}
%******************
%* NEW:
%******************
%\end{codeblock}
{\color{insertcolor}

%    template <class Lock, class Predicate>
%      bool wait_until(Lock& lock,
%                      Predicate pred,
%                      stop_token stoken);
%    template <class Lock, class Clock, class Duration, class Predicate>
%      bool wait_until(Lock& lock,
%                      const chrono::time_point<Clock, Duration>& abs_time
%                      Predicate pred,
%                      stop_token stoken);
%    template <class Lock, class Rep, class Period, class Predicate>
%      bool wait_for(Lock& lock,
%                    const chrono::duration<Rep, Period>& rel_time,
%                    Predicate pred,
%                    stop_token stoken);

%**************************************
\rSec3[thread.condvarany.interruptwait]{Interruptable waits}

The following functions ensure to get notified
if a stop is requested for the passed \tcode{stop_token}.
In that case they return
(returning \tcode{false} if the predicate evaluates to \tcode{false}). 
\begin{note} 
        Because all signatures here call
        \tcode{stop_requested()}, their calls synchronize with \tcode{request_stop()}.
      \end{note}

% Hans Boehm:
%We then also need some statement for the waiting functions that if they return with an stopped/interrupted status,
% their synchronization behavior is as though they called stop_requested().
%{\color{blue}
%[{\itshape{}Editorial note:} Because all signatures here in the effects clause call
%        \tcode{stop_requested()}, we don't need wording
%        that the calls synchronize with \tcode{interrupt()/request_stop()}. ]
%}


%**** untimed wait_until (with pred)
% return false if stoken.stop_requested():

\begin{itemdecl}
template <class Lock, class Predicate>
  bool wait_until(Lock& lock,
                  Predicate pred,
                  stop_token stoken);
\end{itemdecl}
%{\color{blue}
%[{\itshape{}Editorial note:} {\color{diffcolor}This color signals differences to the corresponding \tcode{wait()} function without the stop token parameter.} ]
%}

\begin{itemdescr}

%{\color{diffcolor}
 \pnum\effects Registers \tcode{*this} to get notified when a stop is requested on \tcode{stoken}
                during this call and then equivalent to:
\begin{codeblock}
while(!pred() && !stoken.stop_requested()) {
  wait(lock, [&pred, &stoken] {
                return pred() || stoken.stop_requested();
             });
}
return pred();
\end{codeblock}

 \pnum \begin{note} The returned value indicates whether the predicate evaluated to
        \tcode{true} regardless of whether
        a stop was requested. \end{note}

 \pnum \postconditions Exception or \tcode{lock} is locked by the calling thread.

 \pnum \remarks
        If the function fails to meet the postcondition, \tcode{terminate()}
        shall be called\iref{except.terminate}.
        \begin{note} This can happen if the re-locking of the mutex throws an exception. \end{note}

 \pnum \throws 
        \tcode{std::bad_alloc} if memory for the internal data structures could not be allocated, or
        any exception thrown by \tcode{pred}.

%}%diffcolor

\end{itemdescr}



%**** wait_until (with pred)
% return false if stoken.stop_requested():

\begin{itemdecl}
template <class Lock, class Clock, class Duration, class Predicate>
  bool wait_until(Lock& lock,
                  const chrono::time_point<Clock, Duration>& abs_time
                  Predicate pred,
                  stop_token stoken);
\end{itemdecl}
%{\color{blue}
%[{\itshape{}Editorial note:} {\color{diffcolor}This color signals differences to the corresponding \tcode{wait_until()} function without the stop token parameter.} ]
%}

\begin{itemdescr}
 \pnum\effects Registers \tcode{*this} to get notified when a stop is requested on \tcode{stoken}
                during this call and then equivalent to:
\begin{codeblock}
while(!pred() && !stoken.stop_requested() && Clock::now() < abs_time) {
  cv.wait_until(lock,
                abs_time,
                [&pred, &stoken] {
                  return pred() || stoken.stop_requested();
                });
}
return pred();
\end{codeblock}

\pnum
\begin{note} There is no blocking, if \tcode{pred()} is initially \tcode{true}, 
stoken is not stoppable, a stop was already requested, or
the timeout has already expired. \end{note}

\pnum
\begin{note} The returned value indicates whether the predicate evaluates to \tcode{true}
regardless of whether the timeout was triggered. \end{note}

 \pnum \begin{note} The returned value indicates whether the predicate evaluated to
        \tcode{true} regardless of whether the timeout was triggered
        or a stop was requested. \end{note}

 \pnum \postconditions Exception or \tcode{lock} is locked by the calling thread.

 \pnum \remarks
        If the function fails to meet the postcondition, \tcode{terminate()}
        shall be called\iref{except.terminate}.
        \begin{note} This can happen if the re-locking of the mutex throws an exception. \end{note}

 \pnum \throws 
        \tcode{std::bad_alloc} if memory for the internal data structures could not be allocated,
        any timeout-related exception\iref{thread.req.timing},
        or any exception thrown by \tcode{pred}.

%{\color{diffcolor}
% \pnum\sync If the function returns with an stopped/interrupted status, 
%                their synchronization behavior is as though it called \tcode{stop_requested()}.
%}%diffcolor
\end{itemdescr}


%**** wait_for (with pred)
% return false if stoken.stop_requested():

\begin{itemdecl}
template <class Lock, class Rep, class Period, class Predicate>
  bool wait_for(Lock& lock,
                const chrono::duration<Rep, Period>& rel_time,
                Predicate pred,
                stop_token stoken);
\end{itemdecl}
%{\color{blue}
%[{\itshape{}Editorial note:} {\color{diffcolor}This color signals differences to the corresponding \tcode{wait_for()} function without the stop token parameter.} ]
%}
\begin{itemdescr}
 \pnum \effects Equivalent to:
\begin{codeblock}
return wait_until(lock, chrono::steady_clock::now() + rel_time, std::move(pred), std::move(stoken));
\end{codeblock}
\end{itemdescr}

}%\color{insertcolor}


\vspace{5ex}

%******************************************************************
\rSec1[futures]{Futures}

...

